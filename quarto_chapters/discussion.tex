% Options for packages loaded elsewhere
\PassOptionsToPackage{unicode}{hyperref}
\PassOptionsToPackage{hyphens}{url}
\PassOptionsToPackage{dvipsnames,svgnames,x11names}{xcolor}
%
\documentclass[
]{article}

\usepackage{amsmath,amssymb}
\usepackage{iftex}
\ifPDFTeX
  \usepackage[T1]{fontenc}
  \usepackage[utf8]{inputenc}
  \usepackage{textcomp} % provide euro and other symbols
\else % if luatex or xetex
  \usepackage{unicode-math}
  \defaultfontfeatures{Scale=MatchLowercase}
  \defaultfontfeatures[\rmfamily]{Ligatures=TeX,Scale=1}
\fi
\usepackage{lmodern}
\ifPDFTeX\else  
    % xetex/luatex font selection
\fi
% Use upquote if available, for straight quotes in verbatim environments
\IfFileExists{upquote.sty}{\usepackage{upquote}}{}
\IfFileExists{microtype.sty}{% use microtype if available
  \usepackage[]{microtype}
  \UseMicrotypeSet[protrusion]{basicmath} % disable protrusion for tt fonts
}{}
\makeatletter
\@ifundefined{KOMAClassName}{% if non-KOMA class
  \IfFileExists{parskip.sty}{%
    \usepackage{parskip}
  }{% else
    \setlength{\parindent}{0pt}
    \setlength{\parskip}{6pt plus 2pt minus 1pt}}
}{% if KOMA class
  \KOMAoptions{parskip=half}}
\makeatother
\usepackage{xcolor}
\usepackage[lmargin=20mm,rmargin=20mm]{geometry}
\setlength{\emergencystretch}{3em} % prevent overfull lines
\setcounter{secnumdepth}{5}
% Make \paragraph and \subparagraph free-standing
\makeatletter
\ifx\paragraph\undefined\else
  \let\oldparagraph\paragraph
  \renewcommand{\paragraph}{
    \@ifstar
      \xxxParagraphStar
      \xxxParagraphNoStar
  }
  \newcommand{\xxxParagraphStar}[1]{\oldparagraph*{#1}\mbox{}}
  \newcommand{\xxxParagraphNoStar}[1]{\oldparagraph{#1}\mbox{}}
\fi
\ifx\subparagraph\undefined\else
  \let\oldsubparagraph\subparagraph
  \renewcommand{\subparagraph}{
    \@ifstar
      \xxxSubParagraphStar
      \xxxSubParagraphNoStar
  }
  \newcommand{\xxxSubParagraphStar}[1]{\oldsubparagraph*{#1}\mbox{}}
  \newcommand{\xxxSubParagraphNoStar}[1]{\oldsubparagraph{#1}\mbox{}}
\fi
\makeatother


\providecommand{\tightlist}{%
  \setlength{\itemsep}{0pt}\setlength{\parskip}{0pt}}\usepackage{longtable,booktabs,array}
\usepackage{calc} % for calculating minipage widths
% Correct order of tables after \paragraph or \subparagraph
\usepackage{etoolbox}
\makeatletter
\patchcmd\longtable{\par}{\if@noskipsec\mbox{}\fi\par}{}{}
\makeatother
% Allow footnotes in longtable head/foot
\IfFileExists{footnotehyper.sty}{\usepackage{footnotehyper}}{\usepackage{footnote}}
\makesavenoteenv{longtable}
\usepackage{graphicx}
\makeatletter
\def\maxwidth{\ifdim\Gin@nat@width>\linewidth\linewidth\else\Gin@nat@width\fi}
\def\maxheight{\ifdim\Gin@nat@height>\textheight\textheight\else\Gin@nat@height\fi}
\makeatother
% Scale images if necessary, so that they will not overflow the page
% margins by default, and it is still possible to overwrite the defaults
% using explicit options in \includegraphics[width, height, ...]{}
\setkeys{Gin}{width=\maxwidth,height=\maxheight,keepaspectratio}
% Set default figure placement to htbp
\makeatletter
\def\fps@figure{htbp}
\makeatother
% definitions for citeproc citations
\NewDocumentCommand\citeproctext{}{}
\NewDocumentCommand\citeproc{mm}{%
  \begingroup\def\citeproctext{#2}\cite{#1}\endgroup}
\makeatletter
 % allow citations to break across lines
 \let\@cite@ofmt\@firstofone
 % avoid brackets around text for \cite:
 \def\@biblabel#1{}
 \def\@cite#1#2{{#1\if@tempswa , #2\fi}}
\makeatother
\newlength{\cslhangindent}
\setlength{\cslhangindent}{1.5em}
\newlength{\csllabelwidth}
\setlength{\csllabelwidth}{3em}
\newenvironment{CSLReferences}[2] % #1 hanging-indent, #2 entry-spacing
 {\begin{list}{}{%
  \setlength{\itemindent}{0pt}
  \setlength{\leftmargin}{0pt}
  \setlength{\parsep}{0pt}
  % turn on hanging indent if param 1 is 1
  \ifodd #1
   \setlength{\leftmargin}{\cslhangindent}
   \setlength{\itemindent}{-1\cslhangindent}
  \fi
  % set entry spacing
  \setlength{\itemsep}{#2\baselineskip}}}
 {\end{list}}
\usepackage{calc}
\newcommand{\CSLBlock}[1]{\hfill\break\parbox[t]{\linewidth}{\strut\ignorespaces#1\strut}}
\newcommand{\CSLLeftMargin}[1]{\parbox[t]{\csllabelwidth}{\strut#1\strut}}
\newcommand{\CSLRightInline}[1]{\parbox[t]{\linewidth - \csllabelwidth}{\strut#1\strut}}
\newcommand{\CSLIndent}[1]{\hspace{\cslhangindent}#1}

\usepackage{lineno}\linenumbers
\makeatletter
\@ifpackageloaded{caption}{}{\usepackage{caption}}
\AtBeginDocument{%
\ifdefined\contentsname
  \renewcommand*\contentsname{Table of contents}
\else
  \newcommand\contentsname{Table of contents}
\fi
\ifdefined\listfigurename
  \renewcommand*\listfigurename{List of Figures}
\else
  \newcommand\listfigurename{List of Figures}
\fi
\ifdefined\listtablename
  \renewcommand*\listtablename{List of Tables}
\else
  \newcommand\listtablename{List of Tables}
\fi
\ifdefined\figurename
  \renewcommand*\figurename{Figure}
\else
  \newcommand\figurename{Figure}
\fi
\ifdefined\tablename
  \renewcommand*\tablename{Table}
\else
  \newcommand\tablename{Table}
\fi
}
\@ifpackageloaded{float}{}{\usepackage{float}}
\floatstyle{ruled}
\@ifundefined{c@chapter}{\newfloat{codelisting}{h}{lop}}{\newfloat{codelisting}{h}{lop}[chapter]}
\floatname{codelisting}{Listing}
\newcommand*\listoflistings{\listof{codelisting}{List of Listings}}
\makeatother
\makeatletter
\makeatother
\makeatletter
\@ifpackageloaded{caption}{}{\usepackage{caption}}
\@ifpackageloaded{subcaption}{}{\usepackage{subcaption}}
\makeatother

\ifLuaTeX
  \usepackage{selnolig}  % disable illegal ligatures
\fi
\usepackage{bookmark}

\IfFileExists{xurl.sty}{\usepackage{xurl}}{} % add URL line breaks if available
\urlstyle{same} % disable monospaced font for URLs
\hypersetup{
  pdftitle={Discussion},
  colorlinks=true,
  linkcolor={blue},
  filecolor={Maroon},
  citecolor={Blue},
  urlcolor={Blue},
  pdfcreator={LaTeX via pandoc}}


\title{Discussion}
\author{}
\date{}

\begin{document}
\maketitle


\section{Introduction}\label{introduction}

Hastened genetic progress in potato is dependent on the continued
development and application of quantitative genetics methodology in
potato breeding. Like many vegetable crops, the application of
statistical genetic theory has only received significant attention the
past two decades in potato. Before this, complex trait improvement in
conventional potato breeding was reliant on phenotypic selection and
marked by several inefficiencies including underpowered field
evaluations and long breeding cycles\textsuperscript{{[}1{]}}. With the
continued application of technologies that have had positive success in
other field crops, increased genetic improvement in potato is possible.
These include both methods such as variance component estimation,
genomic prediction, selection index theory as well as the importing of
other breeding schemas like recurrent selection or line breeding.

In this thesis, I consider the conversion of clonal tetraploid potato to
a diploid hybrid crop as an exercise of applied biometrical methods. To
our knowledge, this thesis is the first to interrogate the genetic
architecture of a large F1 crossing block in diploid hybrid potato.
Moreover, these past experimental chapters also took the opportunity to
assess the adequacy of multiple statistical methods in the estimation
and prediction of genetic value in this novel crop. I first examined the
genetical properties of the F1 hybrid seedling trials including genetic
correlations between multiple tuber yield components (chapter
2;\textsuperscript{{[}2{]}}). I also utilized multiple forms of genetic
information to assess the family history of the inbred parental lines.
This included the use of pedigree and molecular marker information
(chapter 3;\textsuperscript{{[}3{]}}) as well as identity by descent
profiles (chapter 5) to evaluate the population structure of these
parental lines. After having estimated these genetic parameters, I
tested multiple predictive modelling methodologies testing the potential
benefits of predicting dominance effects (chapter
3;\textsuperscript{{[}3{]}}), the utility of marker assisted selection
for complex traits (chapter 4), and various multiallelic
parameterisations for genome-wide regression (chapter 5). These topics
cover an important range of topics for modern selection methods in a
potato breeding program.

Here I provide a synthesis of the previous four experimental chapters
with two major aims. I first wish to identify the most important
findings of this research, and second, outline the direct implications
of our research on potato breeding more generally. I then conclude this
thesis with reflections on future research.

\subsection{Biometrical analysis of field
trials}\label{biometrical-analysis-of-field-trials}

A core activity in breeding programs is thoe regular evaluation of new
parental combinations to screen for genetically improved offspring.
Identification of suitable and informative trialling locations ensures
an unbiased and representative assessment of a candidate's performance.
For this reason, within-trial and multi-environmental trial design is so
heavily emphasized in modern crop literature for sufficient statistical
power and optimal resource allocation\textsuperscript{{[}4{]}}. This
permits proper extraction of spatial and other non-genetic components in
a field trial and unbiased evaluation of genotype performance. From this
successful spatial analysis, it was then possible to measure the genetic
parameters of interest whether it be comparison of genotype means, or
the estimation of genetic correlations between traits.

In conventional potato breeding, formal designed trials are reserved for
the 3\textsuperscript{rd} or 4\textsuperscript{th} clonal generations
when enough clonal propagule can guarantee sufficient
replication\textsuperscript{{[}5,6{]}}. This is in contrast to other
field crops where true seed is the predominant source of germplasm and
allows for suitable replication after the initial cross or
selfing\textsuperscript{{[}7{]}}. An overlooked property of hybrid
breeding in potato is the potential of leveraging true seed and the
ability to conduct trialing on multi-plant plots over traditional
tuber-sown hills\textsuperscript{{[}8,9{]}}. The latter tend to be
underpowered with too low of precision for early selection making true
potato seed a powerful resource in breeding germplasm
evaluation\textsuperscript{{[}10,11{]}}. This is not without negative
consequences, chief among them, being the poor correlation in multiple
tuber traits between the seedling and clonal
generations\textsuperscript{{[}12--14{]}}. This is addressed in full in
Section~\ref{sec-implications}, but suffice it to say that the field
evaluation of hybrid potato via seedling transplants provides
opportunities for explicit field design with sufficiently high precision
for multiple tuber phenotypes\textsuperscript{{[}15{]}}.

In \textbf{chapter 2}, such an exercise was conducted in the form of a
multi-trial analysis on several hundred F1 hybrids. A penalized splines
based procedure was used for spatially de-trending our hybrid
observations for three tuber phenotypes. These de-trended phenotypes
were then placed in a multivariate linear mixed model for the estimation
of the genetic (co-)variances for the GCA, SCA, and several interactive
effects. In \textbf{chapter 3}, this was followed up with a traditional
linear mixed modelling procedure where multiple row column designs with
various residual structures were tested on an augmented
dataset\textsuperscript{{[}16{]}}. The BLUEs and standard errors from
these models where then used in a full genomic model where genomically
estimated BLUPs were evaluated in a cross-validation schema. While the
spatial modelling procedures from \textbf{chapter 2} and \textbf{chapter
3} were derived from two different frameworks, little to no difference
was found between the use of semi-parametric Spline functions and
structuring our residual variance. Previous studies have also suggested
these approaches as equivalent with exception of the ease of a single
model-fitting over the testing of multiple linear mixed
models\textsuperscript{{[}17{]}}. Both modelling procedures from these
chapters follow the general form of a two-stage mixed model analysis
where trials are independently analyzed followed by an combined analysis
on the genotypic means (most commonly) together with weights on those
genotypic observations\textsuperscript{{[}18{]}}. Two-stage approaches
are understood to be generally less accurate than than their single
stage counterparts especially with greater model
complexity\textsuperscript{{[}19{]}}. All this being said, single-stage
approaches tend to be unwieldy in their specification and computation
and the consequences of a two-stage approach are unlikely to change any
inferential conclusions from these two
chapters\textsuperscript{{[}20{]}}.

A core concept in hybrid breeding is the ability to assess the potential
of hybrid crosses on the basis of parental value. In the case for
phenotypic selection, general combining abilities tend to be the primary
mode which communicates a line's value\textsuperscript{{[}21{]}}. In
\textbf{chapter 2}, GCAs were generated for 400 parental lines in
multiple tuber traits and found that the GCAs sufficiently captured
between 59 and 71 per cent of the observed phenotypic variation in the
F1 screening trials. The use of GCA for selection is not unique to
hybrid crops with many examples of their use in tetraploid potato in
many modern applications as well\textsuperscript{{[}22,23{]}}. This is
despite many tetraploid studies reporting a large proportion of
non-additive genetic effects controlling tuber qualities like tuber
size, tuber yield, and marketable yield\textsuperscript{{[}24--26{]}}.

From \textbf{chapter 2}, some insight around the genetic architecture of
yield components was gleaned in the form of genetic correlations. This
was not just expressed in terms of a single genetic component over the
hybrids, but explicitly in terms of the additive and non-additive
genetic covariance (\(\Sigma_G = 2 \cdot \Sigma_{gca} + \Sigma_{sca}\)).
This is worth re-emphasis. Potato has many traits are relevant for
selection; many of them being complex traits with low heritabilities and
interdependent relationships\textsuperscript{{[}27{]}}. Because of this,
it is critical for breeders to better utilize multivariate methods in
genetic improvement in potato. In addition, hybrid breeding schemas are
also augmented by multivariate methods whether based upon phenotype
alone or also for multivariate genomic prediction
applications\textsuperscript{{[}28{]}}.

\subsection{Genomic prediction in hybrid
potato}\label{genomic-prediction-in-hybrid-potato}

Genomic prediction has emerged as a cornerstone of modern crop breeding,
offering unprecedented precision in estimating genetic merit by
leveraging genome-wide marker data. By modelling molecular markers with
phenotypic records, genomic prediction enables breeders to identify
superior parental combinations and hybrid crosses with greater accuracy
and efficiency than traditional pedigree-based methods. This section
explores the application, advantages, and limitations of genomic
prediction in hybrid potato breeding, drawing on insights from our
experimental chapters to explore its role in accelerating genetic gain
in potato.

Multiple chapters in this thesis addressed genomic prediction
applications in hybrid potato. In \textbf{chapter 3}, the genetic models
partitioned into GCA and SCA from \textbf{chapter 2} were built upon
with incorporated molecular marker information to structure the each
genetic component. Multiple genomic models were then tested in a
predictive application testing a simple and full genetic model (GCA, and
GCA+SCA, respectively) for the same tuber variates from \textbf{chapter
2} along with tuber dry matter content. Testing the predictive model
performance between each model showed no additional benefit through the
addition of the SCA component in the model. Contrasting the SCA variance
relative to the genetic residual suggested that there were other genetic
effects that were not captured in either of the other genetic effects,
most notably in total tuber number and dry matter content. Through this
modelling schema, it was confirmed that genomic prediction solely on the
basis of GCA sufficed in the estimation of a hybrid cross genetic for
all tuber variates studied.

These prediction models were extended further in \textbf{chapter 5} by
examining several other statistical paradigms and assessing any benefits
to multiallelic marker information. Specific attention was taken to
examining the utility of identity-by-descent (IBD) information derived
from deep pedigree information linking ancestral founders to the
parent's of hybrids along with multiallelic identity-by-state (IBS)
information. Both types of marker information were compared with
conventional biallelic SNPs using traditional shrinkage-based models
along with more complicated kernel prediction. For all tuber variates,
the SNP based prediction models were superior than their multiallelic
counterparts. Marked differences in prediction accuracy were especially
observed between the SNP and IBD models. Similarly, little differences
between the different modelling methods were found with exception that
the Gaussian kernel tended to maximize prediction accuracy regardless of
trait and markerset. These results would suggest that in the context of
genomic prediction, simpler marker parameterizations tend to yield more
consistent prediction outcomes. Despite these results, multi-allelic
marker information are likely to continue to have a prominent role in
inferential applications such as multi-parental population (MPP)
mapping\textsuperscript{{[}29{]}}.

\subsection{Methods of selection}\label{methods-of-selection}

Modern breeding programs are met with unbounding choice with regard to
different technologies. This is especially true for molecular marker and
marker-based methods of selection. Marker-assisted selection (MAS) and
other marker-based techniques have revolutionized modern breeding
programs by enabling precise selection of desirable traits at early
stages of development. In the context of hybrid potato breeding, these
methods leverage molecular markers to identify and select parental lines
with favorable alleles, thereby accelerating genetic gain.

The topic of selection strategy in hybrid potato was inspected in
\textbf{chapter 4}. The exercise comprised of comparing several
molecular marker-assisted strategies based upon their relative
efficiency which primarily depended on the prediction accuracy of the
underlying model used. The primary aim was to asses whether
marker-assisted selection could be as efficient as the genome-wide
regression models utilized in \textbf{chapter 3}. Three strategies were
considered: a marker-assisted selection strategy (\texttt{MA}),
genome-wide prediction (\texttt{GW}), and a single-SNP control strategy
(\texttt{PC}). Each of these were used to predict the value of an inbred
parent. Based upon a forward-selection procedure, 33 unique QTL were
found across three traits (with total tuber number being excluded).
These QTL were able to pick up between 54 and 56 per cent of the total
genetic variation and were the basis for prediction in the models used
in \texttt{MA}. When examining each strategy's selection accuracy, the
\texttt{GW} strategy was evidently superior for all three traits with
the relative efficiency of \texttt{MA} being between 0.89 to 0.95
relative to \texttt{GW}. Because these strategies have different costs
associated with them, primarily related to genotyping cost, these were
also considered. When the genotyping costs were integrated into the
relative efficiency, the \texttt{MA} appeared more favourable.

\section{Wider Implications}\label{wider-implications}

\subsection{Hybrid Breeding Schema}\label{hybrid-breeding-schema}

Two important questions broached in \textbf{chapter's 2 \& 3} dealt with
the nature of gene action in hybrid potato and what population and
breeding strategy should be leveraged to effectively breed for complex
trait improvement. Both of these questions impact the future of what
\emph{kind} of hybrid crop potato might be. While the scope of this
thesis is limited by the relatively narrow genetic background sampled to
initialize these inbred populations\textsuperscript{{[}30{]}}, these
results can still be taken into consideration to inform strategy for
hybrid potato breeding.

One of the major findings \textbf{chapter 2} was a distinct lack of SCA
variance found among the panel of 806 hybrids. This was further
confirmed in \textbf{chapter 3} in a genomic model were the SCA variance
was smaller than both the GCA and genetic residual variance components.
As remarked in these chapters, this would indicate a lack of
non-additive gene action at work among our panel of F1 hybrids. This
could be characteristic of very little population structure among the
inbreds (as confirmed by the population-based analyses in
\textbf{chapter 3}), however, this explanation is not wholly
satisfactory. It has been observed in other hybrid crops that SCA
variance tends to be more present in complex trait architecture where
heterotic pools are not genetically distinct\textsuperscript{{[}31{]}}.
Contrary to this, my results follow the more general pattern seen in
other hybrid crops where the dominance variance appears to be nothing
more than a genetic residual variance with most of the genetic variation
being allocated to the additive variance\textsuperscript{{[}32{]}}. As a
complimentary argument, there are also many examples of higher-order
genic interactions which manifest in statistical models as additive
genetic variance\textsuperscript{{[}33,34{]}}. This too could be
contributing to what was observed earlier in both phenotypic analyses
and subsequent predictive modelling. It is worth acknowledging that this
whole discussion hinges on the assumption of a direct relationship of
estimated statistical and genetic parameters derived
therefrom\textsuperscript{{[}35{]}}. It is out of scope to defend this
premise of applied genetics here, but there are many examples of such
genetic parameters being accurate enough to guide decision-making in
breeding applications\textsuperscript{{[}36,37{]}}. In other words,
``\emph{While these {[}assumptions{]} are formally unrealistic, methods
work.}''\textsuperscript{{[}38{]}}.

Perhaps most pertinent to the discussion of hybrid breeding is with
regard to pool structure and the expediency of multiple pools in potato.
The development of pools in crops like maize or sorghum was a relatively
unguided process where complementation between distinct genetic groups
was first observed often with pedigree-breeding, and then further
developed with more formal methods of population
improvement\textsuperscript{{[}39{]}}. Hybrid crops are primarily the
product of a multi-pool system, however, this is dependent upon multiple
conditions related both to complex trait improvement and the additional
costs of logistics in multi-pool breeding. Multiple simulation studies
have examined the topic of heterotic pool development and are worth
consideration for potato. Simulation of multiple tetraploid and diploid
breeding program designs found that a two pool strategy based around
GCA-based selection was effective in an inbred-hybrid program. However,
this was dependent on the trait's degree of dominance in addition to
hybrid schemas being more capital intensive\textsuperscript{{[}40{]}}.
The authors also remarked on the sensitivity of this strategy not only
to cycle length (especially if reliant on phenotypic-based screening),
but also the time required to generate fully inbred pools.

The question of how pools should be generated is also open for
deliberation. In terms of quantitative trait improvement, multiple
\emph{splitting} strategies have been suggested to be equivalent
including even random pool assignment, at least among selfing
crops\textsuperscript{{[}41{]}}. A more important consideration than
genetic differentiation among pools in potato will be the market segment
requirements. These will strongly govern pool development around trait
targets which are widely discordant. For example, it would be likely
cumbersome to produce parental lines for ware and quick service markets
from the same pool based upon the diverging product profiles of
each\textsuperscript{{[}42{]}}. Rather than seeing heterotic pooling as
some necessity, it is instead the working out of a comprehensive
strategy dependent on the robust production of inbred lines, strong
fertility characteristics, and a reliable cytoplasmic male sterility
mechanism\textsuperscript{{[}43,44{]}}. These are the core factors which
are decisive to the choice of single, two, or other multi-pool breeding
schemas. The topic of fertility is considered further in
Section~\ref{sec-future}.

\subsection{Multivariate applications for potato
breeders}\label{sec-implications}

Multivariate methods have already proven to benefit animal and crop
breeding by enabling simultaneous improvement of multiple traits while
delineating their genetic, environmental, and residual correlations.
These approaches are particularly valuable in potato breeding, where
traits such as tuber yield, tuber quality parameters, and disease
resistance often exhibit complex interdependencies. By integrating
genetic covariances, breeders can design selection indices that maximize
genetic gain across traits, thereby optimizing resource allocation and
accelerating progress toward breeding objectives. While these methods
have been widely adopted in other field crops, their application in
potato remains under-explored.

To lend some credence to multivariate selection in potato, let us
consider what multivariate selection would involve using the trait
genetic covariances reported from \textbf{chapter 2} (Figure
\ref{fig:gca-coef-full-pairs}) including estimated GCA covariance
matrices together with their full phenotypic variance matrix
(\(\mathrm P\)). Assuming the GCA variances estimated here would be
roughly equivalent to those estimated from a test cross schema in a
hybrid breeding program (\(\mathrm {G} \approx \mathrm {V}_{gca}\)), the
selection response on inbred parents could be estimated for future
inbred development using the multivariate breeder's equation as
expressed by\textsuperscript{{[}45{]}}. If we performed truncation
selection on average tuber volume (\(s_{tv}~=~2~cm^3\)) with no direct
selection on total tuber yield and tuber number constant
(\(s_{ty}~=~0~Tonnes~\cdot~Ha^{-1}\) \(s_{tn}~=~0~Tubers\)), and
conducted inter-mating between selected candidates, then the expected
selection response (\(\mathrm R\)) next test cross cycle could be
estimated as:

\[ \mathrm {R = G \cdot P^{-1} \cdot S}\]
\[ \mathrm {R} = \begin{bmatrix}11.54 & 7.27 & 4.95 \\ 7.27 & 11.13 & 67.2 \\ 4.95 & 67.2 & 629.51\end{bmatrix}\cdot\begin{bmatrix}0.17 & -0.22 & 0.02 \\ -0.22 & 0.34 & -0.03 \\ 0.02 & -0.03 & 0\end{bmatrix} \cdot \begin{bmatrix} 2 \\ 0 \\ 0 \end{bmatrix} \]
\[\mathrm {R} = \begin{bmatrix}1.09 \\ 0.64 \\ 0.07\end{bmatrix}\]

This would indicate an increase of 1.1 \(cm^3\) in tuber volume next
test cross cycle with a minor increase in total tuber yield and little
change in tuber number. Note that not only has this increased our
selection response when used over the univariate alternative (0.9
\(= h^2 s\)), but we are also able to evaluate the impact of indirect
selection among the other tuber variates. This is invaluable both in
forecasting and breeding strategy development. More attention is needed
here to make these methods more practical to wield as well as scalable
in applied settings. With regard to practicality, selection indices are
one of the primary tools used to reduce a breeder's multi-dimensional
trait space into into a singular index with a variety of methods
proposed\textsuperscript{{[}46,47{]}}. Speaking to implementability, the
most challenging step in this process is the estimation of the trait
genetic covariance. This tends to become quite unwieldy using
traditional linear mixed modeling methods in higher-dimensional spaces
making other estimation methodology more
attractive\textsuperscript{{[}48,49{]}}.

This same exercise has utility in other facets of potato breeding. One
recurrent hurdle in potato evaluation is the lack of genetic correlation
between the initial seedling stages and subsequent clonal
generations\textsuperscript{{[}50{]}}. These differences are not
restricted to tubers from these generations but are present earlier with
major differences in plant architecture suggesting altered sink-source
dynamics between generations\textsuperscript{{[}51{]}}. This is true
both for phenotypic evaluation and genetic parameter estimation and all
this significantly constrains early selection efficiency within potato
breeding programs\textsuperscript{{[}8,52{]}}. Multivariate estimation
of a candidate's \emph{genetic value} (relevant in clonal programs) or
\emph{breeding value} (relevant for both clonal and hybrid breeding
schemas) jointly across multiple generations could be a valuable
extension of multivariate selection. The estimation of these
inter-generational covariance structures together with a with
traditional multivariate methods could potentially augment forecasting
the potential of a varietal candidate earlier in selection cycles.

\subsection{Successful genomic prediction in hybrid
potato}\label{successful-genomic-prediction-in-hybrid-potato}

Genomic prediction has already accelerated genetic progress in multiple
hybrid crops\textsuperscript{{[}53{]}}. \textbf{Chapter's 3, 4, and 5}
have several implications for genomic prediction applications in potato
as a hybrid crop. Considering factors related to parameterization of
genetic model or type of modelling framework (what I will call more
generally, \emph{model choice}), it is not the \emph{decisive} factor in
the outcome of genomic prediction strategy. Having reviewed extensions
of the traditional GBLUP, various shrinkage-based estimation methods,
and one implementation of the Gaussian kernel, similar performance was
observed with only marginal improvement between models. This is in
keeping with many similar studies in other row crops as well as observed
in tetraploid potato\textsuperscript{{[}54--58{]}}. Whether the model is
an extension of GBLUP or a new member of the Bayesian alphabet, there is
no clear or distinct advantage observed for most traits.

As documented in other crops, often more important than the specific
genetic parameterisations of a model is the actual composition of its
training set\textsuperscript{{[}59{]}}. In many breeding programs,
training sets often arise from the breeding material itself utilizing
previous breeding cycles to estimate the population genetic covariance
or marker effects for a genomic model. Within hybrid breeding schemas,
training sets are frequently developed from test cross blocks which
often have some factorial structure (e.g.~\textbf{t} testers by
\textbf{x} new candidates) or sparse
modification\textsuperscript{{[}60{]}}. These are used both for the
selection of novel lines within pools as well as for the prediction of
hybrid crosses. \textbf{Chapter's 3, 4, and 5} all utilized four field
trials which utilized just under 800 F1 hybrids which were the progeny
of 456 inbred parents in a sparse mating design. While it could be
demonstrated in an earlier chapter that the genetic models based upon
these training sets were technically
\emph{identifiable}\textsuperscript{{[}61{]}}, the sparsity of the
crossing block likely led to bias in the estimation of our genetic
variances and of any predictions of the genetic
effects\textsuperscript{{[}62{]}}. Despite this, our training set still
enabled genomic prediction of hybrid performance and can likely be
improved with further optimization. Aside from incorporating future
cycles to augment the training set, there are many other methods
relevant to increasing the phenotypic and genetic variance in a training
set\textsuperscript{{[}63--65{]}}.

Related to training set composition is the topic of molecular marker
information density. These questions are often pursued along the lines
of predictive benefits in genomic modelling, the hope being that all
genetic variation can be interrogated by some perfect markerset. While
this cannot be fully achieved\textsuperscript{{[}66{]}}, markerset
composition is a worthwhile question in efficient selection
applications. In \textbf{Chapter's 4}, I considered a procedure for
comparing classical marker-assisted selection over genomic prediction
for several quantitative traits. One of the primary conclusions of this
research was that there are multiple complex traits in potato which
could be affectively selected for with only a handful of molecular
markers. This schema was relatively simple and could be augmented
through a more robust simulation of costs, genotype-by-environment
effects, and trial design\textsuperscript{{[}67{]}}. A point should also
be made regarding our higher marker densities, that being, these were
still quite low. Many predictive modelling studies have used marker
densities in the magnitude of hundreds of thousands or million of
molecular markers\textsuperscript{{[}68,69{]}}. Having said this, even
tetraploid studies have shown that small marker panels are still capable
of yielding acceptable results in genomic prediction and other trait
discovery applications which is in keeping with what I found in my
experimental chapters\textsuperscript{{[}70,71{]}}.

Aside from marker density, molecular marker content has captured the
attention of many geneticists. Whether it be the inclusion of functional
genetic information, enrichment of major QTL, or multi-omic data, all
seek to include biologically relevant information to bolster genetic
applications\textsuperscript{{[}72--74{]}}. In this vein,
\textbf{Chapter 5} attempted genomic prediction based upon multiallelic
predictors through the inclusion of haplotag and IBD profiles.
Strategies for the development of multiallelic probe design and genetic
models have become a topic of interest in potato as an answer to
potato's genomic diversity\textsuperscript{{[}75{]}}. While recent
pan-genome studies have shown less haplotype diversity among released
potato varieties then expected (particularly among European cultivars),
potential applications for multiallelic molecular data are still to be
identified\textsuperscript{{[}76{]}}. Statistical models developed with
the haplotag probeset showed nearly identical performance with models
using traditional SNP-based predictors. These results would suggest
little benefit in the inclusion of multiallelic predictors for
genome-wide prediction based upon my survey of models used here. The
performance of the IBD models were notably lower than expectation, but
as noted already, this could be the product of high uncertainty for a
subset of ancestral founders. Whether or not IBD information can extend
genomic prediction applications, it is worthwhile to reiterate that
founder composition in a breeding program can be utilized in
pre-breeding efforts, understanding selection outcomes, and genetic
diversity management.

\section{Future research}\label{sec-future}

There are many facets of potato breeding that have not been sufficiently
addressed by current quantitative genetic frameworks. Here future
applications of quantitative genetics are considered with particular
emphasis on their role in improved fertility, pre-breeding methodology,
and breeding risk assessment for more robust potato breeding.

\subsection{Fertility Mechanisms}\label{fertility-mechanisms}

Fertility is a crucial factor for any seed-based crop. The success of
hybrid breeding systems are particularly dependent on seed cost price
which is chiefly the product of affordable and reliable seed production
systems\textsuperscript{{[}77,78{]}}. Most research in potato fertility
up until this time has \emph{rightly} been focussed on large effect loci
like \emph{Sli} or important genes in wider fertility modules like
\emph{StCDF1}\textsuperscript{{[}79--81{]}}. As these loci continue to
be used and fixed in breeding populations, an important future step will
be to survey a broader array of relevant traits in diploid potato and
assess them beyond QTL\textsuperscript{{[}82{]}}. The typical targets
for fertility in hybrid crops are pollen shed, sufficient pollen
viability, and synced male and female flower opening, such that seed
production is unencumbered\textsuperscript{{[}13,83{]}}. In conjunction
with these traits, a viable cytoplasmic male sterility (CMS) system is
also crucial for F1 hybrid seed production. Not only does this aid in
circumventing female line emasculation costs \& minimizing F1 seed
contamination, but also reduces berry set in the F1 field crop thus
eliminating seed-borne volunteering post harvest. Recently, a CMS system
has been proposed for hybrid potato using the \emph{antherless} locus,
\emph{Al}, on chromosome 6 with male sterility expressed in germplasm
containing the P cytoplasm\textsuperscript{{[}44{]}}. Hybrids with P
cytoplasm and homozygous at the \emph{antherless} with \emph{alal}
showed significantly lower rates of berry set and less seed yields
relative to fertile hybrids. One operational concern with such a
strategy is that it requires not only a single locus to be fixed in both
maternal and paternal lines, but all maternal lines must also contain
the P cytoplasm. It should also be noted that while there is not a clear
elucidation of cytoplasm's exact role on agronomic and economic traits,
there is evidence of cytoplasm affecting maturity, starch content, and
late blight\textsuperscript{{[}84{]}}. This would have serious
ramifications if transferring a P cytoplasm to a breeding program's
elite maternal lines has an associated \emph{plastid-drag} synonymous to
the use of QTLs contained in large introgression regions. Further
validation of the CMS model can help assess cytoplasm substitution's
risk on other economic traits in hybrid potato.

\subsection{Pre-breeding for Quantitative
Improvement}\label{pre-breeding-for-quantitative-improvement}

The role of pre-breeding is often neglected with regard to complex trait
improvement of potato. Pre-breeding strategies in diploid potato are
particularly challenging as you are frequently interested in novel
material with a higher ploidy. Pre-breeding has at least three problems:
(1) How new germplasm should be screened, (2) How should ploidy
reduction be done, and (3) How should that material be incorporated into
a breeding program. None of these questions are very straightforward
with regard to a diploid potato breeding program and is decorated by
multiple hazards. In a more familiar crop breeding scenario, such as
maize, if a new program is being initiated, evaluating a base population
of landraces before introgressing into your breeding program is a
sensible strategy for trait improvement\textsuperscript{{[}85{]}}.
However, in diploid populations, novel tetraploid germplasm must first
be subjected to angiogenesis or \emph{prickle pollination} before
dihaploids can be introduced into a
program\textsuperscript{{[}86,87{]}}. This process not only disrupts the
original trait architecture of that original tetraploid donor, but these
dihaploids frequently bear little to no resemblance to the original
donor making their original evaluation dubious.

Multiple screening approaches have been suggested, all of which, depend
on molecular marker information of some form. One approach proposed in
banana and potato suggests an extension of GBLUP whereby multi-ploidy
training sets could enable prediction from the 4x to 2x ploidy
levels\textsuperscript{{[}88,89{]}}. The prediction accuracy observed in
these studies appear modest, but whether the endeavour of dedicated
tetraploid field evaluation for an augmented across-ploidy training set
is altogether justified is unclear. To avoid this, the IBD estimation
methodology from \textbf{chapter 5} could be considered as a potential
methodology for \emph{linking} the genetic value of dihaploids to their
tetraploid forbearer's. IBD estimation of deep pedigrees in diploids has
been realized in both simulated populations as well as
real\textsuperscript{{[}90{]}}. Additionally, IBD estimation has been
conducted in shallow structured populations in
tetraploids\textsuperscript{{[}91,92{]}}. Very little methodological
development would be needed to extend the IBD tracing from tetraploid to
related dihaploids enabling the estimation of allelic effects (in
diploid populations) from specific tetraploid haplotypes. If a novel
donor has already been introgressed into a program, IBD profiles can
leveraged and even extended to related tetraploids.

Both these methods essentially seek to explain the performance of
diploids by some molecular proxy in the tetraploid either through a
genetic covariance structure or via explicit IBD probabilities. One last
consideration for future improvements in pre-breeding is captured in
what is often called the next stage of breeding or \emph{``Breeding
4.0''}\textsuperscript{{[}93{]}}. The central premise is that a crop's
\emph{ideotype} can be designed from the ground up based upon a library
of known functionally annotated haplotypes which are then combined into
an optimal cultivar via cis or transgenic
technologies\textsuperscript{{[}94{]}}. Breeding is no longer a practice
of finding the proverbial needle in the haystack, but is instead the
direct forging of a variety from constituent components. This would make
the role of pre-breeding primarily focused on identifying those
components. Recent endeavours in diploid potato breeding could be
described as examples of such crop design, most
notably,\textsuperscript{{[}95{]}}. Populations are screened on the
basis of functional genetic load based upon some algorithmic method
(e.g.~SIFT), and are then used to hasten the process of inbred
production. Whether this approach can produce a sufficient volume of
inbreds to initialize a breeding program is to be seen, but is a tool
which is already being used with the intent to design future potato
cultivars.

\subsection{Formalization of Risk}\label{formalization-of-risk}

Often, while considering a novel technology or more efficient breeding
schema, the first question asked is \emph{``What kind of genetic gains
can be expected with this implementation''}? This in of itself is not
unsound, but without asking \emph{``what are the risks of such an
implementation''} biometricians run the risk of overly optimistic
forecasting. This addendum is especially pertinent for potato breeders
due to many of the aforementioned challenges, chief among them,
environmental sensitivity\textsuperscript{{[}96{]}}. Formalizing the
estimation of explicit risk of a breeding strategy is the province of
quantitative genetics and thus deserves greater attention. For example,
decreasing the cycle length of a schema by reducing the number of years
of trial evaluation is a classic route used for increasing genetic gain.
However, this also stands to increase the uncertainty around a candidate
if they have been exposed to an insufficient number of environments.
Embracing other tools might also bring risk to the forefront. Factor
analytic and random regression models have been lauded for their ability
to assess genotype stability and adaptability, and offer incredible
potential for evaluation in potato\textsuperscript{{[}97,98{]}}. Looking
to methods utilized outside of plant breeding such as the use of hazard
models could also be useful for assessing the long-term stability of a
potential potato candidate\textsuperscript{{[}99{]}}.

\subsection{Conclusion}\label{conclusion}

I present here a first examination into the quantitative nature of
several complex traits in hybrid potato. Additionally, through the
survey of a variety of statistical models, I offer a framework for
evaluating hybrid potato genetic architecture, genotye-by-environment
interaction, and genetic progress. This thesis has also demonstrated the
transformative potential of biometrical methods in advancing hybrid
potato breeding. Our findings underscore the importance of building on
traditional statistical formulations for both inferential and predictive
applications in potato breeding. These insights not only bridge critical
gaps in potato breeding but also align with broader trends in crop
improvement, where quantitative genetics and genomic technologies are
reshaping traditional paradigms. Moving forward, the integration of
these methods into breeding programs promises to enhance precision,
reduce cycle times, and unlock the full potential of hybrid potato as a
sustainable crop for future agriculture.

\section*{References}\label{references}
\addcontentsline{toc}{section}{References}

\phantomsection\label{refs}
\begin{CSLReferences}{0}{1}
\bibitem[\citeproctext]{ref-Bradshaw2017}
\CSLLeftMargin{1. }%
\CSLRightInline{Bradshaw JE. Review and {Analysis} of {Limitations} in
{Ways} to {Improve Conventional Potato Breeding}. Potato Research
{[}Internet{]} 2017 {[}cited 2021 Jun 28{]};60:171--93. Available from:
\url{http://www.tsl.ac.uk/news/new-potato-at-the-}}

\bibitem[\citeproctext]{ref-Adams2022}
\CSLLeftMargin{2. }%
\CSLRightInline{Adams JR, de Vries ME, Zheng C, van Eeuwijk FA. Little
heterosis found in diploid hybrid potato: {The} genetic underpinnings of
a new hybrid crop. G3: Genes, Genomes, Genetics {[}Internet{]}
2022;12(6). Available from:
\url{https://www.ncbi.nlm.nih.gov/pubmed/35460241}}

\bibitem[\citeproctext]{ref-Adams2023}
\CSLLeftMargin{3. }%
\CSLRightInline{Adams J, De Vries M, Van Eeuwijk F. Efficient {Genomic
Prediction} of {Yield} and {Dry Matter} in {Hybrid Potato}. Plants
{[}Internet{]} 2023 {[}cited 2023 Nov 9{]};12(14):2617. Available from:
\url{https://www.mdpi.com/2223-7747/12/14/2617}}

\bibitem[\citeproctext]{ref-Schoemaker2024}
\CSLLeftMargin{4. }%
\CSLRightInline{Schoemaker DL, Lima DC, de Leon N, Kaeppler SM. Modeling
the impact of resource allocation decisions on genomic prediction using
maize multi-environment data. Crop Science {[}Internet{]} 2024 {[}cited
2025 Sep 6{]};64(5):2748--67. Available from:
\url{https://onlinelibrary.wiley.com/doi/abs/10.1002/csc2.21305}}

\bibitem[\citeproctext]{ref-Gopal2015}
\CSLLeftMargin{5. }%
\CSLRightInline{Gopal J. Challenges and {Way-forward} in {Selection} of
{Superior Parents}, {Crosses} and {Clones} in {Potato Breeding}. Potato
Research {[}Internet{]} 2015 {[}cited 2023 Jan 20{]};58(2):165--88.
Available from:
\url{https://link-springer-com.ezproxy.library.wur.nl/article/10.1007/s11540-015-9292-6}}

\bibitem[\citeproctext]{ref-Paget2017}
\CSLLeftMargin{6. }%
\CSLRightInline{Paget MF, Alspach PA, Anderson JA, D, Genet RA, Braam
WF, et al. Replicate allocation to improve selection efficiency in the
early stages of a potato breeding scheme. Euphytica {[}Internet{]} 2017
{[}cited 2024 Nov 21{]};213(9):1--15. Available from:
\url{https://www.proquest.com/docview/1934680784/abstract/3B29004E4B074DC0PQ/1}}

\bibitem[\citeproctext]{ref-Zystro2018}
\CSLLeftMargin{7. }%
\CSLRightInline{Zystro J, Colley M, Dawson J. Alternative {Experimental
Designs} for {Plant Breeding} {[}Internet{]}. In: Plant {Breeding
Reviews}. Wiley; 2018 {[}cited 2020 Jan 4{]}. page 87--117.Available
from:
\url{https://onlinelibrary.wiley.com/doi/abs/10.1002/9781119521358.ch3}}

\bibitem[\citeproctext]{ref-Davies1974}
\CSLLeftMargin{8. }%
\CSLRightInline{Davies HT, Johnston GR. Reliability of potato selection
in the first clonal generation. American Potato Journal {[}Internet{]}
1974 {[}cited 2024 Nov 21{]};51(1):8--11. Available from:
\url{https://doi.org/10.1007/BF02852022}}

\bibitem[\citeproctext]{ref-Pallais1991}
\CSLLeftMargin{9. }%
\CSLRightInline{Pallais N. True {Potato Seed}: {Changing Potato
Propagation} from {Vegetative} to {Sexual}. HortScience
1991;26(3):239--41. }

\bibitem[\citeproctext]{ref-Storck2011}
\CSLLeftMargin{10. }%
\CSLRightInline{Storck L, Lopes SJ, Lúcio AD, Cargnelutti Filho A.
Optimum plot size and number of replications related to selective
precision. Cienc Rural {[}Internet{]} 2011 {[}cited 2025 Sep
6{]};41:390--6. Available from:
\url{https://www.scielo.br/j/cr/a/3T8B3TxTLGQXZqPdssPKHRD/?format=html&lang=en}}

\bibitem[\citeproctext]{ref-Stockem2021}
\CSLLeftMargin{11. }%
\CSLRightInline{Stockem JE, Korontzis G, Wilson SE, Vries ME de, Eeuwijk
FA, Struik PC. Optimal plot dimensions for performance testingfof hybrid
potato in the field. Potato Research 2021;}

\bibitem[\citeproctext]{ref-Anderson1981}
\CSLLeftMargin{12. }%
\CSLRightInline{Anderson JaD, Howard HW. Effectiveness of selection in
the early stages of potato breeding programmes. Potato Res
{[}Internet{]} 1981 {[}cited 2025 May 27{]};24(3, 3):289--99. Available
from:
\url{https://link-springer-com.ezproxy.library.wur.nl/article/10.1007/BF02360366}}

\bibitem[\citeproctext]{ref-Ortiz1997}
\CSLLeftMargin{13. }%
\CSLRightInline{Ortiz R. Breeding for potato production from true seed.
1997 {[}cited 2024 Nov 20{]};Available from:
\url{https://www.cabidigitallibrary.org/doi/full/10.5555/19971610712}}

\bibitem[\citeproctext]{ref-Gu2024}
\CSLLeftMargin{14. }%
\CSLRightInline{Gu J, Struik PC, Evers JB, Lertngim N, Lin R, Driever
SM. Quantifying differences in plant architectural development between
hybrid potato ({Solanum} tuberosum) plants grown from two types of
propagules. Annals of Botany {[}Internet{]} 2024 {[}cited 2025 Nov
2{]};133(2):365--78. Available from:
\url{https://academic.oup.com/aob/article-abstract/133/2/365/7473623}}

\bibitem[\citeproctext]{ref-VanDijk2021a}
\CSLLeftMargin{15. }%
\CSLRightInline{van Dijk LCM, Lommen WJM, de Vries ME, Kacheyo OC,
Struik PC. Hilling of {Transplanted Seedlings} from {Novel Hybrid True
Potato Seeds Does Not Enhance Tuber Yield} but {Can Affect Tuber Size
Distribution}. Potato Res {[}Internet{]} 2021 {[}cited 2025 May
28{]};64(3, 3):353--74. Available from:
\url{https://link.springer.com/article/10.1007/s11540-020-09481-x}}

\bibitem[\citeproctext]{ref-Gilmour1997}
\CSLLeftMargin{16. }%
\CSLRightInline{Gilmour AR, Cullis BR, Verbyla AP. Accounting for
{Natural} and {Extraneous Variation} in the {Analysis} of {Field
Experiments}. Journal of Agricultural, Biological, and Environmental
Statistics {[}Internet{]} 1997 {[}cited 2025 May 28{]};2(3):269--93.
Available from: \url{https://www.jstor.org/stable/1400446}}

\bibitem[\citeproctext]{ref-Velazco2017}
\CSLLeftMargin{17. }%
\CSLRightInline{Velazco JG, Rodríguez-Álvarez MX, Boer MP, Jordan DR,
Eilers PHC, Malosetti M, et al. Modelling spatial trends in sorghum
breeding field trials using a two-dimensional {P-spline} mixed model.
Theoretical and Applied Genetics {[}Internet{]} 2017 {[}cited 2021 Jun
8{]};130(7):1375--92. Available from: \url{https://cran.r-project.org/}}

\bibitem[\citeproctext]{ref-Piepho2012}
\CSLLeftMargin{18. }%
\CSLRightInline{Piepho HP, Möhring J, Schulz-Streeck T, Ogutu JO. A
stage-wise approach for the analysis of multi-environment trials.
Biometrical Journal {[}Internet{]} 2012 {[}cited 2025 May
28{]};54(6):844--60. Available from:
\url{https://onlinelibrary.wiley.com/doi/abs/10.1002/bimj.201100219}}

\bibitem[\citeproctext]{ref-Gogel2018}
\CSLLeftMargin{19. }%
\CSLRightInline{Gogel B, Smith A, Cullis B. Comparison of a one- and
two-stage mixed model analysis of {Australia}'s {National Variety Trial
Southern Region} wheat data. Euphytica {[}Internet{]} 2018 {[}cited 2025
May 28{]};214(2, 2):1--21. Available from:
\url{https://link.springer.com/article/10.1007/s10681-018-2116-4}}

\bibitem[\citeproctext]{ref-Damesa2017}
\CSLLeftMargin{20. }%
\CSLRightInline{Damesa TM, Möhring J, Worku M, Piepho HP. One {Step} at
a {Time}: {Stage-Wise Analysis} of a {Series} of {Experiments}. Agronomy
Journal {[}Internet{]} 2017 {[}cited 2023 Feb 17{]};109(3):845--57.
Available from:
\url{https://acsess-onlinelibrary-wiley-com.ezproxy.library.wur.nl/doi/10.2134/agronj2016.07.0395}}

\bibitem[\citeproctext]{ref-Bertan2007}
\CSLLeftMargin{21. }%
\CSLRightInline{Bertan I, de Carvalho FI, Oliveira AC de. Parental
selection strategies in plant breeding programs. Journal of crop science
and biotechnology {[}Internet{]} 2007 {[}cited 2025 Jun
11{]};10(4):211--22. Available from:
\url{https://koreascience.kr/article/JAKO200716637994242.page}}

\bibitem[\citeproctext]{ref-Neele1991}
\CSLLeftMargin{22. }%
\CSLRightInline{Neele AEF, Nab HJ, Louwes KM. Identification of superior
parents in a potato breeding programme. Theoretical and Applied Genetics
{[}Internet{]} 1991 {[}cited 2021 Oct 9{]};82(3):264--72. Available
from: \url{https://link.springer.com/article/10.1007/BF02190611}}

\bibitem[\citeproctext]{ref-Bradshaw2003}
\CSLLeftMargin{23. }%
\CSLRightInline{Bradshaw JE, Dale ·MFB, Mackay ·GR.
\href{https://doi.org/10.1007/s00122-003-1219-y}{Use of mid-parent
values and progeny tests to increase the efficiency of potato breeding
for combined processing quality and disease and pest resistance}. Theor
Appl Genet 2003;107:36--42. }

\bibitem[\citeproctext]{ref-Plaisted1962}
\CSLLeftMargin{24. }%
\CSLRightInline{Plaisted RL, Sanford L, Federer WT, Kehr AE, Peterson
LC. \href{https://doi.org/10.1007/BF02871402}{Specific and general
combining ability for yield in potatoes}. American Potato Journal
1962;39(5):185--97. }

\bibitem[\citeproctext]{ref-Tai1976}
\CSLLeftMargin{25. }%
\CSLRightInline{Tai GCC. {ESTIMATION OF GENERAL AND SPECIFIC COMBINING
ABILITIES IN POTATO}. Can J Genet Cytol {[}Internet{]} 1976 {[}cited
2025 May 17{]};18(3):463--70. Available from:
\url{http://www.nrcresearchpress.com/doi/10.1139/g76-056}}

\bibitem[\citeproctext]{ref-Killick1977}
\CSLLeftMargin{26. }%
\CSLRightInline{Killick RJ. Genetic analysis of several traits in
potatoes by means of a diallel cross. Annals of Applied Biology
{[}Internet{]} 1977 {[}cited 2021 Oct 9{]};86(2):279--89. Available
from:
\url{https://onlinelibrary-wiley-com.ezproxy.library.wur.nl/doi/full/10.1111/j.1744-7348.1977.tb01841.x}}

\bibitem[\citeproctext]{ref-Gebhardt2013}
\CSLLeftMargin{27. }%
\CSLRightInline{Gebhardt C. Bridging the gap between genome analysis and
precision breeding in potato. Trends in Genetics {[}Internet{]} 2013
{[}cited 2023 Feb 1{]};29(4):248--56. Available from:
\url{https://www.sciencedirect.com/science/article/pii/S0168952512001904}}

\bibitem[\citeproctext]{ref-Wang2017}
\CSLLeftMargin{28. }%
\CSLRightInline{Wang X, Li L, Yang Z, Zheng X, Yu S, Xu C, et al.
Predicting rice hybrid performance using univariate and multivariate
{GBLUP} models based on {North Carolina} mating design {II}. Heredity
{[}Internet{]} 2017 {[}cited 2025 May 28{]};118(3):302--10. Available
from: \url{https://www.nature.com/articles/hdy201687}}

\bibitem[\citeproctext]{ref-Li2021a}
\CSLLeftMargin{29. }%
\CSLRightInline{Li W, Boer MP, Zheng C, Joosen RVL, van Eeuwijk FA. An
{IBD-based} mixed model approach for {QTL} mapping in multiparental
populations. Theoretical and Applied Genetics 2021 134:11 {[}Internet{]}
2021 {[}cited 2021 Nov 1{]};134(11):3643--60. Available from:
\url{https://link.springer.com/article/10.1007/s00122-021-03919-7}}

\bibitem[\citeproctext]{ref-Lindhout2018}
\CSLLeftMargin{30. }%
\CSLRightInline{Lindhout P, de Vries M, ter Maat M, Ying S,
Viquez-Zamora Marcela, van Deusden S.
\href{https://doi.org/10.1201/9781351114455-10/HYBRID-POTATO-BREEDING-IMPROVED-VARIETIES-GEFU-WANG-PRUSKI}{Hybrid
potato breeding for improved varieties}. In: Wang-Pruski Gefu, editor.
Achieving sustainable cultivation of potatoes {Volume} 1. Burleigh Dodds
Science Publishing; 2018. page 119--42.}

\bibitem[\citeproctext]{ref-Zhao2015b}
\CSLLeftMargin{31. }%
\CSLRightInline{Zhao Y, Mette MF, Reif JC. Genomic selection in hybrid
breeding. Plant Breeding {[}Internet{]} 2015 {[}cited 2023 May
1{]};134(1):1--10. Available from:
\url{https://onlinelibrary.wiley.com/doi/abs/10.1111/pbr.12231}}

\bibitem[\citeproctext]{ref-Bernardo2016}
\CSLLeftMargin{32. }%
\CSLRightInline{Bernardo R. Bandwagons {I}, too, have known. Theoretical
and Applied Genetics {[}Internet{]} 2016;129(12):2323--32. Available
from: \url{https://www.ncbi.nlm.nih.gov/pubmed/27681088}}

\bibitem[\citeproctext]{ref-Hill2008}
\CSLLeftMargin{33. }%
\CSLRightInline{Hill WG, Goddard ME, Visscher PM. Data and {Theory
Point} to {Mainly Additive Genetic Variance} for {Complex Traits}. PLOS
Genetics {[}Internet{]} 2008 {[}cited 2025 Oct 2{]};4(2):e1000008.
Available from:
\url{https://journals.plos.org/plosgenetics/article?id=10.1371/journal.pgen.1000008}}

\bibitem[\citeproctext]{ref-Monnahan2015}
\CSLLeftMargin{34. }%
\CSLRightInline{Monnahan PJ, Kelly JK. Epistasis {Is} a {Major
Determinant} of the {Additive Genetic Variance} in {Mimulus} guttatus.
PLOS Genetics {[}Internet{]} 2015 {[}cited 2025 Oct 2{]};11(5):e1005201.
Available from:
\url{https://journals.plos.org/plosgenetics/article?id=10.1371/journal.pgen.1005201}}

\bibitem[\citeproctext]{ref-Campos2015}
\CSLLeftMargin{35. }%
\CSLRightInline{Campos G de los, Sorensen D, Gianola D. Genomic
{Heritability}: {What Is It}? PLOS Genetics {[}Internet{]} 2015 {[}cited
2025 Sep 27{]};11(5):e1005048. Available from:
\url{https://journals.plos.org/plosgenetics/article?id=10.1371/journal.pgen.1005048}}

\bibitem[\citeproctext]{ref-Walsh2004}
\CSLLeftMargin{36. }%
\CSLRightInline{Walsh B. Population-and quantitative-genetic models of
selection limits. Plant breeding reviews {[}Internet{]} 2004 {[}cited
2025 Oct 2{]};24(1):177--226. Available from:
\url{https://scholar.archive.org/work/galt5pvqqvdvvaku2o7mrtuf74/access/wayback/http://nitro.biosci.arizona.edu:80/workshops/Aarhus2006/pdfs/WalshLT.pdf}}

\bibitem[\citeproctext]{ref-Daetwyler2013}
\CSLLeftMargin{37. }%
\CSLRightInline{Daetwyler HD, Calus MPL, Pong-Wong R, de los Campos G,
Hickey JM. Genomic {Prediction} in {Animals} and {Plants}: {Simulation}
of {Data}, {Validation}, {Reporting}, and {Benchmarking}. Genetics
{[}Internet{]} 2013 {[}cited 2025 Oct 2{]};193(2):347--65. Available
from: \url{https://doi.org/10.1534/genetics.112.147983}}

\bibitem[\citeproctext]{ref-Hill2010}
\CSLLeftMargin{38. }%
\CSLRightInline{Hill WG. Understanding and using quantitative genetic
variation. Philosophical Transactions of the Royal Society B: Biological
Sciences {[}Internet{]} 2010 {[}cited 2025 Oct 2{]};365(1537):73--85.
Available from:
\url{https://royalsocietypublishing.org/doi/full/10.1098/rstb.2009.0203}}

\bibitem[\citeproctext]{ref-Duvick2004}
\CSLLeftMargin{39. }%
\CSLRightInline{Duvick DN, Smith JSC, Cooper M. Long-term {Selection} in
a {Commercial Hybrid Maize Breeding Program}. Plant Breeding Reviews
2004;24(2):109--52. }

\bibitem[\citeproctext]{ref-Labroo2023}
\CSLLeftMargin{40. }%
\CSLRightInline{Labroo MR, Endelman JB, Gemenet DC, Werner CR, Gaynor
RC, Covarrubias-Pazaran GE. Clonal diploid and autopolyploid breeding
strategies to harness heterosis: Insights from stochastic simulation.
Theoretical and Applied Genetics {[}Internet{]} 2023 {[}cited 2023 Jun
9{]};136(7):1--19. Available from:
\url{https://link.springer.com/article/10.1007/s00122-023-04377-z}}

\bibitem[\citeproctext]{ref-Cowling2020}
\CSLLeftMargin{41. }%
\CSLRightInline{Cowling WA, Gaynor RC, Antolín R, Gorjanc G, Edwards SM,
Powell O, et al. In silico simulation of future hybrid performance to
evaluate heterotic pool formation in a self-pollinating crop. Scientific
Reports {[}Internet{]} 2020 {[}cited 2023 May 6{]};10(1):4037. Available
from: \url{https://www.nature.com/articles/s41598-020-61031-0}}

\bibitem[\citeproctext]{ref-Keijbets2008}
\CSLLeftMargin{42. }%
\CSLRightInline{Keijbets MJH. Potato {Processing} for the {Consumer}:
{Developments} and {Future Challenges}. Potato Res {[}Internet{]} 2008
{[}cited 2025 Aug 9{]};51(3):271--81. Available from:
\url{https://doi.org/10.1007/s11540-008-9104-3}}

\bibitem[\citeproctext]{ref-McGrath2018}
\CSLLeftMargin{43. }%
\CSLRightInline{McGrath JM, Panella L. Sugar {Beet Breeding}
{[}Internet{]}. In: Goldman I, editor. Plant {Breeding Reviews}. Wiley;
2018 {[}cited 2024 Mar 20{]}. page 167--218.Available from:
\url{https://onlinelibrary.wiley.com/doi/10.1002/9781119521358.ch5}}

\bibitem[\citeproctext]{ref-Eggers2024}
\CSLLeftMargin{44. }%
\CSLRightInline{Eggers EJ, Su Y, van der Poel E, Flipsen M, de Vries ME,
Bachem CWB, et al. Identification, {Elucidation} and {Deployment} of a
{Cytoplasmic Male Sterility System} for {Hybrid Potato}. Biology
{[}Internet{]} 2024 {[}cited 2025 Oct 25{]};13(6):447. Available from:
\url{https://www.mdpi.com/2079-7737/13/6/447}}

\bibitem[\citeproctext]{ref-Lande1983}
\CSLLeftMargin{45. }%
\CSLRightInline{Lande R, Arnold SJ. The {Measurement} of {Selection} on
{Correlated Characters}. Evolution {[}Internet{]} 1983 {[}cited 2025 May
30{]};37(6):1210--26. Available from:
\url{https://www.jstor.org/stable/2408842}}

\bibitem[\citeproctext]{ref-Kempthorne1959}
\CSLLeftMargin{46. }%
\CSLRightInline{Kempthorne O, Nordskog AW. Restricted {Selection
Indices}. Biometrics {[}Internet{]} 1959 {[}cited 2025 May
30{]};15(1):10--9. Available from:
\url{https://www.jstor.org/stable/2527598}}

\bibitem[\citeproctext]{ref-Bulmer1981}
\CSLLeftMargin{47. }%
\CSLRightInline{Bulmer MG. Selection {Indeces}. In: The mathematical
theory of quantitative genetics. Oxford, UK: Clarendon Press; 1981. page
255.}

\bibitem[\citeproctext]{ref-Blows2009}
\CSLLeftMargin{48. }%
\CSLRightInline{Blows M, Walsh B.
\href{https://doi.org/10.1007/978-1-4020-9005-9_6}{Spherical {Cows
Grazing} in {Flatland}: {Constraints} to {Selection} and {Adaptation}}.
In: van der Werf J, Graser HU, Frankham R, Gondro C, editors. Adaptation
and {Fitness} in {Animal Populations}. Dordrecht: Springer; 2009. page
83--101.}

\bibitem[\citeproctext]{ref-Runcie2013}
\CSLLeftMargin{49. }%
\CSLRightInline{Runcie DE, Mukherjee S. Dissecting {High-Dimensional
Phenotypes} with {Bayesian Sparse Factor Analysis} of {Genetic
Covariance Matrices}. Genetics {[}Internet{]} 2013 {[}cited 2025 May
29{]};194(3):753--67. Available from:
\url{https://doi.org/10.1534/genetics.113.151217}}

\bibitem[\citeproctext]{ref-Maris1988}
\CSLLeftMargin{50. }%
\CSLRightInline{Maris B. Correlations within and between characters
between and within generations as a measure for the early generation
selection in potato breeding. Euphytica {[}Internet{]} 1988 {[}cited
2024 Nov 21{]};37(3):205--24. Available from:
\url{https://doi.org/10.1007/BF00015117}}

\bibitem[\citeproctext]{ref-Gu2025}
\CSLLeftMargin{51. }%
\CSLRightInline{Gu J, Evers JB, Struik PC, Driever SM. Architectural
{Differences} between {Hybrid Potato Plants Grown} from {True Seeds} and
{Seedling Tubers Do Not Influence Leaf Photosynthetic Traits}. Potato
Res {[}Internet{]} 2025 {[}cited 2025 Oct 25{]};Available from:
\url{https://doi.org/10.1007/s11540-025-09930-5}}

\bibitem[\citeproctext]{ref-Gopal1998}
\CSLLeftMargin{52. }%
\CSLRightInline{Gopal J. General combining ability and its repeatability
in early generations of potato breeding programmes. Potato Research
{[}Internet{]} 1998 {[}cited 2024 Nov 21{]};41(1):21--8. Available from:
\url{https://www.proquest.com/docview/754151133/abstract/AD7024E9634C42D0PQ/1}}

\bibitem[\citeproctext]{ref-Labroo2021}
\CSLLeftMargin{53. }%
\CSLRightInline{Labroo MR, Studer AJ, Rutkoski JE.
\href{https://doi.org/10.3389/FGENE.2021.643761}{Heterosis and {Hybrid
Crop Breeding}: {A Multidisciplinary Review}}. Frontiers in Genetics
2021;12:234. }

\bibitem[\citeproctext]{ref-Sverrisdottir2017}
\CSLLeftMargin{54. }%
\CSLRightInline{Sverrisdóttir E, Byrne S, Sundmark EHR, Johnsen HØ, Kirk
HG, Asp T, et al. Genomic prediction of starch content and chipping
quality in tetraploid potato using genotyping-by-sequencing. Theor Appl
Genet {[}Internet{]} 2017 {[}cited 2025 Jul 25{]};130(10):2091--108.
Available from: \url{https://doi.org/10.1007/s00122-017-2944-y}}

\bibitem[\citeproctext]{ref-Endelman2018}
\CSLLeftMargin{55. }%
\CSLRightInline{Endelman JB, Carley CAS, Bethke PC, Coombs JJ, Clough
ME, da Silva WL, et al. Genetic {Variance Partitioning} and {Genome-Wide
Prediction} with {Allele Dosage Information} in {Autotetraploid Potato}.
Genetics {[}Internet{]} 2018 {[}cited 2022 Nov 14{]};209(1):77--87.
Available from: \url{https://doi.org/10.1534/genetics.118.300685}}

\bibitem[\citeproctext]{ref-Amadeu2019}
\CSLLeftMargin{56. }%
\CSLRightInline{Amadeu RR, Ferrão LFV, Oliveira I de B, Benevenuto J,
Endelman JB, Munoz PR.
\href{https://doi.org/10.2135/cropsci2019.02.0138}{Impact of {Dominance
Effects} on {Autotetraploid Genomic Prediction}}. Crop Science
2019;0(0):1--9. }

\bibitem[\citeproctext]{ref-Wilson2021a}
\CSLLeftMargin{57. }%
\CSLRightInline{Wilson S, Zheng C, Maliepaard C, Mulder HA, Visser RGF,
van der Burgt A, et al. Understanding the {Effectiveness} of {Genomic
Prediction} in {Tetraploid Potato}. Frontiers in Plant Science
{[}Internet{]} 2021 {[}cited 2022 Nov 15{]};12. Available from:
\url{https://www.frontiersin.org/articles/10.3389/fpls.2021.672417}}

\bibitem[\citeproctext]{ref-Schrauf2021}
\CSLLeftMargin{58. }%
\CSLRightInline{Schrauf MF, de los Campos G, Munilla S. Comparing
{Genomic Prediction Models} by {Means} of {Cross Validation}. Front
Plant Sci {[}Internet{]} 2021 {[}cited 2025 Aug 16{]};12. Available
from:
\url{https://www.frontiersin.org/journals/plant-science/articles/10.3389/fpls.2021.734512/full}}

\bibitem[\citeproctext]{ref-Dias2019}
\CSLLeftMargin{59. }%
\CSLRightInline{Dias KOG, Piepho ·HP, Guimarães ·LJM, Guimarães ·PEO,
Parentoni ·SN, Pinto ·MO, et al. Novel strategies for genomic prediction
of untested single-cross maize hybrids using unbalanced historical data.
Theoretical and Applied Genetics {[}Internet{]} 2019 {[}cited 2019 Dec
4{]};1:3. Available from:
\url{https://doi.org/10.1007/s00122-019-03475-1}}

\bibitem[\citeproctext]{ref-Jarquin2020}
\CSLLeftMargin{60. }%
\CSLRightInline{Jarquin D, Howard R, Crossa J, Beyene Y, Gowda M,
Martini JWR, et al. Genomic {Prediction Enhanced Sparse Testing} for
{Multi-environment Trials}. G3 Genes\textbar Genomes\textbar Genetics
{[}Internet{]} 2020 {[}cited 2025 Jul 25{]};10(8):2725--39. Available
from: \url{https://doi.org/10.1534/g3.120.401349}}

\bibitem[\citeproctext]{ref-Xenakis2019}
\CSLLeftMargin{61. }%
\CSLRightInline{Xenakis J. Special {Topics For Recombinant Inbred
Intercross Data}: {Model Identifiability}, {Hypothesis Testing} and
{Compositional Methods} {[}Internet{]}. 2019 {[}cited 2021 Jul
6{]};Available from:
\url{https://search.proquest.com/openview/28f4f585a836499e5a03e93f37b54d04/1?pq-origsite=gscholar&cbl=18750&diss=y}}

\bibitem[\citeproctext]{ref-Mohring2011}
\CSLLeftMargin{62. }%
\CSLRightInline{Möhring J, Melchinger AE, Piepho HP. {REML-Based Diallel
Analysis}. Crop Science {[}Internet{]} 2011 {[}cited 2021 Feb
3{]};51(2):470--8. Available from:
\url{http://doi.wiley.com/10.2135/cropsci2010.05.0272}}

\bibitem[\citeproctext]{ref-Isidro2015}
\CSLLeftMargin{63. }%
\CSLRightInline{Isidro J, Jannink JL, Akdemir D, Poland J, Heslot N,
Sorrells ME. Training set optimization under population structure in
genomic selection. Theor Appl Genet {[}Internet{]} 2015 {[}cited 2025
Aug 16{]};128(1):145--58. Available from:
\url{https://doi.org/10.1007/s00122-014-2418-4}}

\bibitem[\citeproctext]{ref-Berro2019}
\CSLLeftMargin{64. }%
\CSLRightInline{Berro I, Lado B, Nalin RS, Quincke M, Gutiérrez L.
\href{https://doi.org/10.3835/plantgenome2019.04.0028}{Training
population optimization for genomic selection}. Plant Genome 2019;12(3).
}

\bibitem[\citeproctext]{ref-Ou2019}
\CSLLeftMargin{65. }%
\CSLRightInline{Ou JH, Liao CT. Training set determination for genomic
selection. Theor Appl Genet {[}Internet{]} 2019 {[}cited 2025 Aug
16{]};132(10):2781--92. Available from:
\url{https://doi.org/10.1007/s00122-019-03387-0}}

\bibitem[\citeproctext]{ref-Brachi2011}
\CSLLeftMargin{66. }%
\CSLRightInline{Brachi B, Morris GP, Borevitz JO. Genome-wide
association studies in plants: The missing heritability is in the field.
Genome Biol {[}Internet{]} 2011 {[}cited 2025 Aug 16{]};12(10):232.
Available from: \url{https://doi.org/10.1186/gb-2011-12-10-232}}

\bibitem[\citeproctext]{ref-Peixoto2024}
\CSLLeftMargin{67. }%
\CSLRightInline{Peixoto MA, Coelho IF, Leach KA, Bhering LL, Resende Jr.
MFR. Simulation-based decision-making and implementation of tools in
hybrid crop breeding pipelines. Crop Science {[}Internet{]} 2024
{[}cited 2025 Aug 16{]};64(1):110--25. Available from:
\url{https://onlinelibrary.wiley.com/doi/abs/10.1002/csc2.21139}}

\bibitem[\citeproctext]{ref-Druet2014}
\CSLLeftMargin{68. }%
\CSLRightInline{Druet T, Macleod IM, Hayes BJ. Toward genomic prediction
from whole-genome sequence data: Impact of sequencing design on genotype
imputation and accuracy of predictions. Heredity {[}Internet{]} 2014
{[}cited 2025 Aug 16{]};112(1):39--47. Available from:
\url{https://www.nature.com/articles/hdy201313}}

\bibitem[\citeproctext]{ref-Weber2024}
\CSLLeftMargin{69. }%
\CSLRightInline{Weber SE, Roscher-Ehrig L, Kox T, Abbadi A, Stahl A,
Snowdon RJ. Genomic prediction in {Brassica} napus: Evaluating the
benefit of imputed whole-genome sequencing data. Genome {[}Internet{]}
2024 {[}cited 2025 Aug 16{]};67(7):210--22. Available from:
\url{https://cdnsciencepub.com/doi/full/10.1139/gen-2023-0126}}

\bibitem[\citeproctext]{ref-Aalborg2024}
\CSLLeftMargin{70. }%
\CSLRightInline{Aalborg T, Sverrisdóttir E, Kristensen HT, Nielsen KL.
The effect of marker types and density on genomic prediction and {GWAS}
of key performance traits in tetraploid potato. Front Plant Sci
{[}Internet{]} 2024 {[}cited 2025 Aug 16{]};15. Available from:
\url{https://www.frontiersin.org/journals/plant-science/articles/10.3389/fpls.2024.1340189/full}}

\bibitem[\citeproctext]{ref-Leyva-Perez2022}
\CSLLeftMargin{71. }%
\CSLRightInline{Leyva-Pérez M de la O, Vexler L, Byrne S, Clot CR, Meade
F, Griffin D, et al. {PotatoMASH}---{A Low Cost}, {Genome-Scanning
Marker System} for {Use} in {Potato Genomics} and {Genetics
Applications}. Agronomy {[}Internet{]} 2022 {[}cited 2024 Sep
3{]};12(10):2461. Available from:
\url{https://www.mdpi.com/2073-4395/12/10/2461}}

\bibitem[\citeproctext]{ref-Guo2016}
\CSLLeftMargin{72. }%
\CSLRightInline{Guo Z, Magwire MM, Basten CJ, Xu Z, Wang D. Evaluation
of the utility of gene expression and metabolic information for genomic
prediction in maize. Theor Appl Genet {[}Internet{]} 2016 {[}cited 2025
Aug 16{]};129(12):2413--27. Available from:
\url{https://doi.org/10.1007/s00122-016-2780-5}}

\bibitem[\citeproctext]{ref-MacLeod2016}
\CSLLeftMargin{73. }%
\CSLRightInline{MacLeod IM, Bowman PJ, Vander Jagt CJ, Haile-Mariam M,
Kemper KE, Chamberlain AJ, et al. Exploiting biological priors and
sequence variants enhances {QTL} discovery and genomic prediction of
complex traits. BMC Genomics {[}Internet{]} 2016 {[}cited 2025 Aug
16{]};17(1):144. Available from:
\url{https://doi.org/10.1186/s12864-016-2443-6}}

\bibitem[\citeproctext]{ref-Arouisse2024}
\CSLLeftMargin{74. }%
\CSLRightInline{Arouisse B, Thoen MPM, Kruijer W, Kunst JF, Jongsma MA,
Keurentjes JJB, et al. Bivariate {GWA} mapping reveals associations
between aliphatic glucosinolates and plant responses to thrips and heat
stress. The Plant Journal {[}Internet{]} 2024 {[}cited 2025 Aug
16{]};120(2):674--86. Available from:
\url{https://onlinelibrary.wiley.com/doi/abs/10.1111/tpj.17009}}

\bibitem[\citeproctext]{ref-Tang2022}
\CSLLeftMargin{75. }%
\CSLRightInline{Tang D, Jia Y, Zhang J, Li H, Cheng L, Wang P, et al.
Genome evolution and diversity of wild and cultivated potatoes. Nature
{[}Internet{]} 2022 {[}cited 2025 Jan 3{]};606(7914):535--41. Available
from: \url{https://www.nature.com/articles/s41586-022-04822-x}}

\bibitem[\citeproctext]{ref-Sun2025}
\CSLLeftMargin{76. }%
\CSLRightInline{Sun H, Tusso S, Dent CI, Goel M, Wijfjes RY, Baus LC, et
al. The phased pan-genome of tetraploid {European} potato. Nature
{[}Internet{]} 2025 {[}cited 2025 Aug 16{]};642(8067):389--97. Available
from: \url{https://www.nature.com/articles/s41586-025-08843-0}}

\bibitem[\citeproctext]{ref-Mao1998}
\CSLLeftMargin{77. }%
\CSLRightInline{Mao CX, Virmani SS, Kumar I. Technological innovations
to lower the costs of hybrid rice seed production. In: Siddiq EA,
Muralidharan K, editors. Advances in hybrid rice technology: Proceedings
of the 3prebreedingrd {International Symposium} on {Hybrid Rice}, 14-16
{November} 1996, {Hyderabad}, {India}. Los Baños, Philippines:
International Rice Research Institute; 1998. }

\bibitem[\citeproctext]{ref-Longin2014}
\CSLLeftMargin{78. }%
\CSLRightInline{Longin CFH, Reif JC, Würschum T. Long-term perspective
of hybrid versus line breeding in wheat based on quantitative genetic
theory. Theor Appl Genet {[}Internet{]} 2014 {[}cited 2025 May
31{]};127(7, 7):1635--41. Available from:
\url{https://link.springer.com/article/10.1007/s00122-014-2325-8}}

\bibitem[\citeproctext]{ref-Clot2020}
\CSLLeftMargin{79. }%
\CSLRightInline{Clot CR, Polzer C, Prodhomme C, Schuit C, Engelen CJM,
Hutten RCB, et al. The origin and widespread occurrence of {Sli-based}
self-compatibility in potato. Theor Appl Genet {[}Internet{]} 2020
{[}cited 2025 May 31{]};133(9, 9):2713--28. Available from:
\url{https://link.springer.com/article/10.1007/s00122-020-03627-8}}

\bibitem[\citeproctext]{ref-Eggers2021}
\CSLLeftMargin{80. }%
\CSLRightInline{Eggers EJ, van der Burgt A, van Heusden SAW, de Vries
ME, Visser RGF, Bachem CWB, et al. Neofunctionalisation of the {Sli}
gene leads to self-compatibility and facilitates precision breeding in
potato. Nature Communications {[}Internet{]} 2021 {[}cited 2023 Jan
20{]};12(1):4141. Available from:
\url{https://www.nature.com/articles/s41467-021-24267-6}}

\bibitem[\citeproctext]{ref-Song2022}
\CSLLeftMargin{81. }%
\CSLRightInline{Song L, Endelman J.
\href{https://doi.org/10.1101/2022.11.09.515871}{Using {Haplotype} and
{QTL Analysis} to {Fix Favorable Alleles} in {Diploid Potato Breeding}}.
2022. }

\bibitem[\citeproctext]{ref-Kempe2011}
\CSLLeftMargin{82. }%
\CSLRightInline{Kempe K, Gils M. Pollination control technologies for
hybrid breeding. Mol Breeding {[}Internet{]} 2011 {[}cited 2025 May
31{]};27(4, 4):417--37. Available from:
\url{https://link.springer.com/article/10.1007/s11032-011-9555-0}}

\bibitem[\citeproctext]{ref-Longin2012}
\CSLLeftMargin{83. }%
\CSLRightInline{Longin CFH, Mühleisen J, Maurer HP, Zhang H, Gowda M,
Reif JC. Hybrid breeding in autogamous cereals. Theoretical and Applied
Genetics {[}Internet{]} 2012 {[}cited 2024 Apr 30{]};125(6):1087--96.
Available from:
\url{http://link.springer.com/10.1007/s00122-012-1967-7}}

\bibitem[\citeproctext]{ref-Sanetomo2015}
\CSLLeftMargin{84. }%
\CSLRightInline{Sanetomo R, Gebhardt C. Cytoplasmic genome types of
{European} potatoes and their effects on complex agronomic traits. BMC
Plant Biol {[}Internet{]} 2015 {[}cited 2025 Nov 2{]};15(1):162.
Available from: \url{https://doi.org/10.1186/s12870-015-0545-y}}

\bibitem[\citeproctext]{ref-Gorjanc2016}
\CSLLeftMargin{85. }%
\CSLRightInline{Gorjanc G, Jenko J, Hearne SJ, Hickey JM. Initiating
maize pre-breeding programs using genomic selection to harness polygenic
variation from landrace populations. BMC Genomics {[}Internet{]} 2016
{[}cited 2023 Oct 31{]};17(1):30. Available from:
\url{http://www.biomedcentral.com/1471-2164/17/30}}

\bibitem[\citeproctext]{ref-Uijtewaal1987}
\CSLLeftMargin{86. }%
\CSLRightInline{Uijtewaal BA, Huigen DJ, Hermsen JGTh. Production of
potato monohaploids (2n=x=12) through prickle pollination. Theoret Appl
Genetics {[}Internet{]} 1987 {[}cited 2025 Aug 2{]};73(5):751--8.
Available from: \url{https://doi.org/10.1007/BF00260786}}

\bibitem[\citeproctext]{ref-Tai2003}
\CSLLeftMargin{87. }%
\CSLRightInline{Tai GCC, Xiong XY. Haploid production of potatoes by
anther culture {[}Internet{]}. In: Maluszynski M, Kasha KJ, Forster BP,
Szarejko I, editors. Doubled {Haploid Production} in {Crop Plants}: {A
Manual}. Dordrecht: Springer Netherlands; 2003 {[}cited 2025 Aug 2{]}.
page 229--34.Available from:
\url{https://doi.org/10.1007/978-94-017-1293-4_34}}

\bibitem[\citeproctext]{ref-Nyine2018}
\CSLLeftMargin{88. }%
\CSLRightInline{Nyine M, Uwimana B, Blavet N, Hřibová E, Vanrespaille H,
Batte M, et al. Genomic {Prediction} in a {Multiploid Crop}: {Genotype}
by {Environment Interaction} and {Allele Dosage Effects} on {Predictive
Ability} in {Banana}. The Plant Genome {[}Internet{]} 2018 {[}cited 2025
Sep 13{]};11(2):170090. Available from:
\url{https://onlinelibrary.wiley.com/doi/abs/10.3835/plantgenome2017.10.0090}}

\bibitem[\citeproctext]{ref-Wilson2021}
\CSLLeftMargin{89. }%
\CSLRightInline{Wilson S, Malosetti M, Maliepaard C, Mulder HA, Visser
RGF, van Eeuwijk F. Training {Set Construction} for {Genomic Prediction}
in {Auto-Tetraploids}: {An Example} in {Potato}. Frontiers in Plant
Science {[}Internet{]} 2021 {[}cited 2022 Nov 15{]};12. Available from:
\url{https://www.frontiersin.org/articles/10.3389/fpls.2021.771075}}

\bibitem[\citeproctext]{ref-Li2021}
\CSLLeftMargin{90. }%
\CSLRightInline{Li W, Boer MP, Zheng C, Joosen RVL, van Eeuwijk FA. An
{IBD-based} mixed model approach for {QTL} mapping in multiparental
populations. Theoretical and Applied Genetics 2021 134:11 {[}Internet{]}
2021 {[}cited 2021 Nov 1{]};134(11):3643--60. Available from:
\url{https://link.springer.com/article/10.1007/s00122-021-03919-7}}

\bibitem[\citeproctext]{ref-Amadeu2021}
\CSLLeftMargin{91. }%
\CSLRightInline{Amadeu RR, Muñoz PR, Zheng C, Endelman JB. {QTL} mapping
in outbred tetraploid (and diploid) diallel populations. Genetics
{[}Internet{]} 2021 {[}cited 2023 Aug 22{]};219(3):iyab124. Available
from: \url{https://www.ncbi.nlm.nih.gov/pmc/articles/PMC8570786/}}

\bibitem[\citeproctext]{ref-Song2023}
\CSLLeftMargin{92. }%
\CSLRightInline{Song L, Endelman JB. Using haplotype and {QTL} analysis
to fix favorable alleles in diploid potato breeding. The Plant Genome
{[}Internet{]} 2023 {[}cited 2025 Sep 13{]};16(2):e20339. Available
from:
\url{https://acsess.onlinelibrary.wiley.com/doi/10.1002/tpg2.20339}}

\bibitem[\citeproctext]{ref-Wallace2018}
\CSLLeftMargin{93. }%
\CSLRightInline{Wallace JG, Rodgers-Melnick E, Buckler ES. On the {Road}
to {Breeding} 4.0: {Unraveling} the {Good}, the {Bad}, and the {Boring}
of {Crop Quantitative Genomics}. Annual Review of Genetics
{[}Internet{]} 2018 {[}cited 2025 May 31{]};52:421--44. Available from:
\url{https://www.annualreviews.org/content/journals/10.1146/annurev-genet-120116-024846}}

\bibitem[\citeproctext]{ref-Varshney2021}
\CSLLeftMargin{94. }%
\CSLRightInline{Varshney RK, Bohra A, Yu J, Graner A, Zhang Q, Sorrells
ME. Designing {Future Crops}: {Genomics-Assisted Breeding Comes} of
{Age}. Trends in Plant Science {[}Internet{]} 2021 {[}cited 2025 Sep
13{]};26(6):631--49. Available from:
\url{https://www.cell.com/trends/plant-science/abstract/S1360-1385(21)00069-8}}

\bibitem[\citeproctext]{ref-Zhang2021}
\CSLLeftMargin{95. }%
\CSLRightInline{Zhang C, Yang Z, Tang D, Zhu Y, Wang P, Li D, et al.
\href{https://doi.org/10.1016/J.CELL.2021.06.006}{Genome design of
hybrid potato}. Cell 2021;184(15):3873--3883.e12. }

\bibitem[\citeproctext]{ref-George2017}
\CSLLeftMargin{96. }%
\CSLRightInline{George TS, Taylor MA, Dodd IC, White PJ. Climate
{Change} and {Consequences} for {Potato Production}: A {Review} of
{Tolerance} to {Emerging Abiotic Stress}. Potato Res {[}Internet{]} 2017
{[}cited 2025 May 30{]};60(3, 3):239--68. Available from:
\url{https://link.springer.com/article/10.1007/s11540-018-9366-3}}

\bibitem[\citeproctext]{ref-Souza2020}
\CSLLeftMargin{97. }%
\CSLRightInline{Souza MH de, Júnior JDP, Steckling SDM, Mencalha J, Dias
F dos S, Rocha JR do AS de C, et al. Adaptability and stability analyses
of plants using random regression models. PLOS ONE {[}Internet{]} 2020
{[}cited 2025 May 30{]};15(12):e0233200. Available from:
\url{https://journals.plos.org/plosone/article?id=10.1371/journal.pone.0233200}}

\bibitem[\citeproctext]{ref-Smith2018}
\CSLLeftMargin{98. }%
\CSLRightInline{Smith AB, Cullis BR. Plant breeding selection tools
built on factor analytic mixed models for multi-environment trial data.
Euphytica {[}Internet{]} 2018 {[}cited 2023 Feb 16{]};214(8):143.
Available from: \url{https://doi.org/10.1007/s10681-018-2220-5}}

\bibitem[\citeproctext]{ref-Yazdi2002a}
\CSLLeftMargin{99. }%
\CSLRightInline{Yazdi MH, Visscher PM, Ducrocq V, Thompson R.
Heritability, {Reliability} of {Genetic Evaluations} and {Response} to
{Selection} in {Proportional Hazard Models}. Journal of Dairy Science
{[}Internet{]} 2002 {[}cited 2024 Nov 21{]};85(6):1563--77. Available
from:
\url{https://www.sciencedirect.com/science/article/pii/S0022030202742264}}

\end{CSLReferences}




\end{document}
