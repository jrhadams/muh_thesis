% Options for packages loaded elsewhere
\PassOptionsToPackage{unicode}{hyperref}
\PassOptionsToPackage{hyphens}{url}
\PassOptionsToPackage{dvipsnames,svgnames,x11names}{xcolor}
%
\documentclass[
]{article}

\usepackage{amsmath,amssymb}
\usepackage{iftex}
\ifPDFTeX
  \usepackage[T1]{fontenc}
  \usepackage[utf8]{inputenc}
  \usepackage{textcomp} % provide euro and other symbols
\else % if luatex or xetex
  \usepackage{unicode-math}
  \defaultfontfeatures{Scale=MatchLowercase}
  \defaultfontfeatures[\rmfamily]{Ligatures=TeX,Scale=1}
\fi
\usepackage{lmodern}
\ifPDFTeX\else  
    % xetex/luatex font selection
\fi
% Use upquote if available, for straight quotes in verbatim environments
\IfFileExists{upquote.sty}{\usepackage{upquote}}{}
\IfFileExists{microtype.sty}{% use microtype if available
  \usepackage[]{microtype}
  \UseMicrotypeSet[protrusion]{basicmath} % disable protrusion for tt fonts
}{}
\makeatletter
\@ifundefined{KOMAClassName}{% if non-KOMA class
  \IfFileExists{parskip.sty}{%
    \usepackage{parskip}
  }{% else
    \setlength{\parindent}{0pt}
    \setlength{\parskip}{6pt plus 2pt minus 1pt}}
}{% if KOMA class
  \KOMAoptions{parskip=half}}
\makeatother
\usepackage{xcolor}
\usepackage[lmargin=20mm,rmargin=20mm]{geometry}
\setlength{\emergencystretch}{3em} % prevent overfull lines
\setcounter{secnumdepth}{-\maxdimen} % remove section numbering
% Make \paragraph and \subparagraph free-standing
\makeatletter
\ifx\paragraph\undefined\else
  \let\oldparagraph\paragraph
  \renewcommand{\paragraph}{
    \@ifstar
      \xxxParagraphStar
      \xxxParagraphNoStar
  }
  \newcommand{\xxxParagraphStar}[1]{\oldparagraph*{#1}\mbox{}}
  \newcommand{\xxxParagraphNoStar}[1]{\oldparagraph{#1}\mbox{}}
\fi
\ifx\subparagraph\undefined\else
  \let\oldsubparagraph\subparagraph
  \renewcommand{\subparagraph}{
    \@ifstar
      \xxxSubParagraphStar
      \xxxSubParagraphNoStar
  }
  \newcommand{\xxxSubParagraphStar}[1]{\oldsubparagraph*{#1}\mbox{}}
  \newcommand{\xxxSubParagraphNoStar}[1]{\oldsubparagraph{#1}\mbox{}}
\fi
\makeatother


\providecommand{\tightlist}{%
  \setlength{\itemsep}{0pt}\setlength{\parskip}{0pt}}\usepackage{longtable,booktabs,array}
\usepackage{calc} % for calculating minipage widths
% Correct order of tables after \paragraph or \subparagraph
\usepackage{etoolbox}
\makeatletter
\patchcmd\longtable{\par}{\if@noskipsec\mbox{}\fi\par}{}{}
\makeatother
% Allow footnotes in longtable head/foot
\IfFileExists{footnotehyper.sty}{\usepackage{footnotehyper}}{\usepackage{footnote}}
\makesavenoteenv{longtable}
\usepackage{graphicx}
\makeatletter
\def\maxwidth{\ifdim\Gin@nat@width>\linewidth\linewidth\else\Gin@nat@width\fi}
\def\maxheight{\ifdim\Gin@nat@height>\textheight\textheight\else\Gin@nat@height\fi}
\makeatother
% Scale images if necessary, so that they will not overflow the page
% margins by default, and it is still possible to overwrite the defaults
% using explicit options in \includegraphics[width, height, ...]{}
\setkeys{Gin}{width=\maxwidth,height=\maxheight,keepaspectratio}
% Set default figure placement to htbp
\makeatletter
\def\fps@figure{htbp}
\makeatother

\usepackage{lineno}\linenumbers
\makeatletter
\@ifpackageloaded{caption}{}{\usepackage{caption}}
\AtBeginDocument{%
\ifdefined\contentsname
  \renewcommand*\contentsname{Table of contents}
\else
  \newcommand\contentsname{Table of contents}
\fi
\ifdefined\listfigurename
  \renewcommand*\listfigurename{List of Figures}
\else
  \newcommand\listfigurename{List of Figures}
\fi
\ifdefined\listtablename
  \renewcommand*\listtablename{List of Tables}
\else
  \newcommand\listtablename{List of Tables}
\fi
\ifdefined\figurename
  \renewcommand*\figurename{Figure}
\else
  \newcommand\figurename{Figure}
\fi
\ifdefined\tablename
  \renewcommand*\tablename{Table}
\else
  \newcommand\tablename{Table}
\fi
}
\@ifpackageloaded{float}{}{\usepackage{float}}
\floatstyle{ruled}
\@ifundefined{c@chapter}{\newfloat{codelisting}{h}{lop}}{\newfloat{codelisting}{h}{lop}[chapter]}
\floatname{codelisting}{Listing}
\newcommand*\listoflistings{\listof{codelisting}{List of Listings}}
\makeatother
\makeatletter
\makeatother
\makeatletter
\@ifpackageloaded{caption}{}{\usepackage{caption}}
\@ifpackageloaded{subcaption}{}{\usepackage{subcaption}}
\makeatother

\ifLuaTeX
  \usepackage{selnolig}  % disable illegal ligatures
\fi
\usepackage[style=authoryear,]{biblatex}
\addbibresource{../library.bib}
\usepackage{bookmark}

\IfFileExists{xurl.sty}{\usepackage{xurl}}{} % add URL line breaks if available
\urlstyle{same} % disable monospaced font for URLs
\hypersetup{
  pdftitle={Discussion},
  colorlinks=true,
  linkcolor={blue},
  filecolor={Maroon},
  citecolor={Blue},
  urlcolor={Blue},
  pdfcreator={LaTeX via pandoc}}


\title{Discussion}
\author{}
\date{}

\begin{document}
\maketitle


\section{Outline}\label{outline}

\begin{enumerate}
\def\labelenumi{\arabic{enumi}.}
\item
  Introductory paragraph contextualising the past chapters (1-2
  paragraphs)
\item
  Summarising the past few chapters (5-6 paragraphs)

  \begin{enumerate}
  \def\labelenumii{\roman{enumii})}
  \tightlist
  \item
    Hybrid yield can be adequately represented by their parental
    combinations ( chapter 2)
  \item
    In hybrid potato populations, we can decompose genetic correlations
    into their predominately additive and dominance components (GCA and
    SCA as proxies) (chapter 2)
  \item
    These same models can be overlaid with marker data for genomic
    prediction/selection applications. (chapter 3)
  \item
    Marker data showed little evidence of structure in population.
  \item
    For both phenotypic and genomic models, some traits
  \end{enumerate}
\item
  Implications of these results

  \begin{enumerate}
  \def\labelenumii{\roman{enumii})}
  \tightlist
  \item
    Spatial modelling. Any paradigm should do.
  \item
    SCA is a genetic residual \autocite{Bernardo2016} (chapter 2) A. Not
    something worth trying to predict in diploids. B. Using GCA's for
    selection in young potato populations.

    \begin{enumerate}
    \def\labelenumiii{\Alph{enumiii}.}
    \setcounter{enumiii}{2}
    \tightlist
    \item
    \end{enumerate}
  \item
    Multi-trait models a benefit for potato breeders. (Chapter 2) A.
    Index selection can be of direct use to breeders. B. Technical
    problems applying this in production.
  \item
    Tuber size and dry matter content are better candidates for
    selection than tuber number A. Sensitivity of tuber number. B.
    Larger residual variance
  \item
  \item
    Hybrid breeding requires statistical modelling to evaluate genetic
    variance and trait architecture (chapter 2)

    \begin{enumerate}
    \def\labelenumiii{\Alph{enumiii}.}
    \tightlist
    \item
      Can use this for future selection forecasting
    \item
      Impact of selection on other traits
    \end{enumerate}
  \item
    Technologies like genomic prediction can be applied quite simply

    \begin{enumerate}
    \def\labelenumiii{\Alph{enumiii}.}
    \tightlist
    \item
      Require smaller training set sizes relative to tetraploids
    \item
      You are driving selection of parent development
    \item
      Hybrid prediction for coming cycles
    \end{enumerate}
  \item
    We can evaluate the efficiency of different technologies
  \item
    We can evaluate the efficiency of different models and information
  \end{enumerate}
\item
  Statistical genetic topics critical in potato

  \begin{enumerate}
  \def\labelenumii{\roman{enumii})}
  \tightlist
  \item
    Dealing with low seedling and tuber-sown genetic correlations

    \begin{enumerate}
    \def\labelenumiii{\Alph{enumiii}.}
    \tightlist
    \item
      Touch on seedling versus clonal cropping systems
    \item
      Review literature on lack of correlation
    \item
      Propose early seedling evaluation and multi-trait prediction
      models as potential solution
    \end{enumerate}
  \item
    Evaluating GxE and sensitivity rigorously
  \item
    Germplasm aquisition and evaluation (pre-breeding topics)

    \begin{enumerate}
    \def\labelenumiii{\Alph{enumiii}.}
    \tightlist
    \item
      Siezing ploidy. Effective tetraploid mining for diploid breeding
    \item
      Address other breeding strategies such as bridge breeding
      (Corentin Clot)
    \end{enumerate}
  \end{enumerate}
\item
  Statistical genetic topics crucial in hybrid breeding

  \begin{enumerate}
  \def\labelenumii{\roman{enumii})}
  \tightlist
  \item
    Fertility and seed production in potato inbreds

    \begin{enumerate}
    \def\labelenumiii{\Alph{enumiii}.}
    \tightlist
    \item
      Affordable production
    \item
      inbreeding depression
    \item
      Genetic factors outside \emph{sli}
    \end{enumerate}
  \item
    Genetic transformation

    \begin{enumerate}
    \def\labelenumiii{\Alph{enumiii}.}
    \tightlist
    \item
      The collaborative role of gene-editting in quantitative trait
      improvement
    \item
      The need for regeneration and transformation as traits

      \begin{itemize}
      \tightlist
      \item
        Necessary for Doubled haploids, genetic transformation, and
        double monoploid production
      \item
        Genetic variation in response identified in potato
      \item
        Genes found in other crops \autocite{Koornneef1993}
      \end{itemize}
    \item
      Building elite inducers \autocite{Delzer2024}
    \end{enumerate}
  \item
    Pipeline for new traits for new production systems
  \end{enumerate}
\item
  Wrapping up / Conclusions about hybrid breeding in potato

  \begin{enumerate}
  \def\labelenumii{\roman{enumii})}
  \tightlist
  \item
    Current status of hybrid breeding research in potato
  \item
    This thesis' place in advancing knowledge about hybrid potato
  \end{enumerate}
\end{enumerate}


\printbibliography



\end{document}
