% Options for packages loaded elsewhere
\PassOptionsToPackage{unicode}{hyperref}
\PassOptionsToPackage{hyphens}{url}
\PassOptionsToPackage{dvipsnames,svgnames,x11names}{xcolor}
%
\documentclass[
]{article}

\usepackage{amsmath,amssymb}
\usepackage{iftex}
\ifPDFTeX
  \usepackage[T1]{fontenc}
  \usepackage[utf8]{inputenc}
  \usepackage{textcomp} % provide euro and other symbols
\else % if luatex or xetex
  \usepackage{unicode-math}
  \defaultfontfeatures{Scale=MatchLowercase}
  \defaultfontfeatures[\rmfamily]{Ligatures=TeX,Scale=1}
\fi
\usepackage{lmodern}
\ifPDFTeX\else  
    % xetex/luatex font selection
\fi
% Use upquote if available, for straight quotes in verbatim environments
\IfFileExists{upquote.sty}{\usepackage{upquote}}{}
\IfFileExists{microtype.sty}{% use microtype if available
  \usepackage[]{microtype}
  \UseMicrotypeSet[protrusion]{basicmath} % disable protrusion for tt fonts
}{}
\makeatletter
\@ifundefined{KOMAClassName}{% if non-KOMA class
  \IfFileExists{parskip.sty}{%
    \usepackage{parskip}
  }{% else
    \setlength{\parindent}{0pt}
    \setlength{\parskip}{6pt plus 2pt minus 1pt}}
}{% if KOMA class
  \KOMAoptions{parskip=half}}
\makeatother
\usepackage{xcolor}
\usepackage[lmargin=20mm,rmargin=20mm]{geometry}
\setlength{\emergencystretch}{3em} % prevent overfull lines
\setcounter{secnumdepth}{5}
% Make \paragraph and \subparagraph free-standing
\makeatletter
\ifx\paragraph\undefined\else
  \let\oldparagraph\paragraph
  \renewcommand{\paragraph}{
    \@ifstar
      \xxxParagraphStar
      \xxxParagraphNoStar
  }
  \newcommand{\xxxParagraphStar}[1]{\oldparagraph*{#1}\mbox{}}
  \newcommand{\xxxParagraphNoStar}[1]{\oldparagraph{#1}\mbox{}}
\fi
\ifx\subparagraph\undefined\else
  \let\oldsubparagraph\subparagraph
  \renewcommand{\subparagraph}{
    \@ifstar
      \xxxSubParagraphStar
      \xxxSubParagraphNoStar
  }
  \newcommand{\xxxSubParagraphStar}[1]{\oldsubparagraph*{#1}\mbox{}}
  \newcommand{\xxxSubParagraphNoStar}[1]{\oldsubparagraph{#1}\mbox{}}
\fi
\makeatother


\providecommand{\tightlist}{%
  \setlength{\itemsep}{0pt}\setlength{\parskip}{0pt}}\usepackage{longtable,booktabs,array}
\usepackage{calc} % for calculating minipage widths
% Correct order of tables after \paragraph or \subparagraph
\usepackage{etoolbox}
\makeatletter
\patchcmd\longtable{\par}{\if@noskipsec\mbox{}\fi\par}{}{}
\makeatother
% Allow footnotes in longtable head/foot
\IfFileExists{footnotehyper.sty}{\usepackage{footnotehyper}}{\usepackage{footnote}}
\makesavenoteenv{longtable}
\usepackage{graphicx}
\makeatletter
\def\maxwidth{\ifdim\Gin@nat@width>\linewidth\linewidth\else\Gin@nat@width\fi}
\def\maxheight{\ifdim\Gin@nat@height>\textheight\textheight\else\Gin@nat@height\fi}
\makeatother
% Scale images if necessary, so that they will not overflow the page
% margins by default, and it is still possible to overwrite the defaults
% using explicit options in \includegraphics[width, height, ...]{}
\setkeys{Gin}{width=\maxwidth,height=\maxheight,keepaspectratio}
% Set default figure placement to htbp
\makeatletter
\def\fps@figure{htbp}
\makeatother

\usepackage{lineno}\linenumbers
\makeatletter
\@ifpackageloaded{caption}{}{\usepackage{caption}}
\AtBeginDocument{%
\ifdefined\contentsname
  \renewcommand*\contentsname{Table of contents}
\else
  \newcommand\contentsname{Table of contents}
\fi
\ifdefined\listfigurename
  \renewcommand*\listfigurename{List of Figures}
\else
  \newcommand\listfigurename{List of Figures}
\fi
\ifdefined\listtablename
  \renewcommand*\listtablename{List of Tables}
\else
  \newcommand\listtablename{List of Tables}
\fi
\ifdefined\figurename
  \renewcommand*\figurename{Figure}
\else
  \newcommand\figurename{Figure}
\fi
\ifdefined\tablename
  \renewcommand*\tablename{Table}
\else
  \newcommand\tablename{Table}
\fi
}
\@ifpackageloaded{float}{}{\usepackage{float}}
\floatstyle{ruled}
\@ifundefined{c@chapter}{\newfloat{codelisting}{h}{lop}}{\newfloat{codelisting}{h}{lop}[chapter]}
\floatname{codelisting}{Listing}
\newcommand*\listoflistings{\listof{codelisting}{List of Listings}}
\makeatother
\makeatletter
\makeatother
\makeatletter
\@ifpackageloaded{caption}{}{\usepackage{caption}}
\@ifpackageloaded{subcaption}{}{\usepackage{subcaption}}
\makeatother

\ifLuaTeX
  \usepackage{selnolig}  % disable illegal ligatures
\fi
\usepackage[style=authoryear,]{biblatex}
\addbibresource{../library.bib}
\usepackage{bookmark}

\IfFileExists{xurl.sty}{\usepackage{xurl}}{} % add URL line breaks if available
\urlstyle{same} % disable monospaced font for URLs
\hypersetup{
  pdftitle={Discussion},
  colorlinks=true,
  linkcolor={blue},
  filecolor={Maroon},
  citecolor={Blue},
  urlcolor={Blue},
  pdfcreator={LaTeX via pandoc}}


\title{Discussion}
\author{}
\date{}

\begin{document}
\maketitle


\section{Introduction}\label{introduction}

Hastened genetic progress in potato is dependent on the continued
development and application of quantitative genetics methodology in
potato breeding. Like many vegetable crops, the application of
statistical genetic theory has only received significant attention the
past two decades in potato. Before this, complex trait improvement in
conventional potato breeding was reliant on phenotypic selection and
marked by several inefficiencies including underpowered field
evaluations and long breeding cycles \autocite{Bradshaw2017}. With the
continued application of technologies that have had positive success in
other field crops, increased genetic improvement in potato is possible.
These include both methods such as variance component estimation,
genomic prediction, selection index theory as well as the importing of
other breeding schemas like recurrent selection or line breeding.

In this thesis, we consider the conversion of clonal tetraploid potato
to a diploid hybrid crop as an exercise of applied biometrical methods.
To our knowledge, this thesis is the first to interrogate the genetic
architecture of a large F1 crossing block in diploid hybrid potato.
Moreover, these past experimental chapters also took the opportunity to
assess the adequacy of multiple statistical methods in the estimation
and prediction of genetic value in this novel crop. We first examined
the genetical properties of the F1 hybrid seedling trials including
genetic correlations between multiple tuber yield components (chapter 2;
\textcite{Adams2022}). We also utilized multiple forms of genetic
information to assess the family history of the inbred parental lines.
This included the use of pedigree and molecular marker information
(chapter 3; \textcite{Adams2023}) as well as identity by descent
profiles (chapter 5) to evaluate the population structure of these
parental lines. After having estimated these genetic parameters, we
tested multiple predictive modelling methodologies testing the potential
benefits of predicting dominance effects (chapter 3;
\textcite{Adams2023}), the utility of marker assisted selection for
complex traits (chapter 4), and various multiallelic parameterisations
for genome-wide regression (chapter 5). These topics cover an important
range of topics for modern selection methods in a potato breeding
program.

Here we provide a synthesis of the previous four experimental chapters
with two major aims. We first wish to identify the most important
findings of this research, and second, outline the direct implications
of our research on potato breeding more generally. We then conclude this
thesis with reflections on future research.

\subsection{Biometrical analysis of field
trials}\label{biometrical-analysis-of-field-trials}

A core activity in breeding programs is the regular evaluation of new
parental combinations to screen for genetically improved offspring.
Identification of suitable and informative trialling locations ensures
an unbiased and representative assessment of a candidate's performance.
For this reason, within-trial and multi-environmental trial design is so
heavily emphasized in modern crop literature for sufficient statistical
power and optimal resource allocation \autocite{Shoemaker2024}. This
permits proper extraction of spatial and other non-genetic components in
a field trial and unbiased evaluation of genotype performance. From this
successful spatial analysis, we are then able to measure the genetic
parameters of interest whether it be comparison of genotype means, or
the estimation of genetic correlations between traits.

In conventional potato breeding, formal trial evaluation is
traditionally reserved for the 1st or 2nd clonal generations (C1 and C2,
respectively) when enough clonal propagule can guarantee sufficient
replication \autocite{Gopal2015,Paget2017}. This is in contrast to other
field crops where true seed is the predominant source of germplasm and
allows for suitable replication after the initial cross or selfing
\autocite{Zystro2018}. An overlooked property of hybrid breeding in
potato is the necessity of true seed and the ability to conduct trialing
on multi-plant plots over traditional tuber-sown hills (commonplace in
the first clonal generation with traditional potato)
\autocite{Davies1974,Pallais1991}. The latter tend to be underpowered
with too low of precision for early selection making true potato seed a
powerful resource in breeding germplasm evaluation
\autocite{Storck2011,Stockem2021}. This is not without negative
consequences, chief among them, being the poor correlation in multiple
tuber traits between the seedling and clonal generations
\autocite{Ortiz1997,Anderson1981}. This is addressed in full in
Section~\ref{sec-implications}, but suffice it to say that the field
evaluation of hybrid potato via seedling transplants provides
opportunities for explicit field design with sufficiently high precision
for multiple tuber phenotypes \autocite{VanDijk2021a}.

In \textbf{chapter 2}, such an exercise was conducted in the form of a
multi-trial analysis on several hundred F1 hybrids. A penalized splines
based procedure was used for spatially de-trending our hybrid
observations for three tuber phenotypes. These de-trended phenotypes
were then placed in a multivariate linear mixed model for the estimation
of the genetic (co-)variances for the GCA, SCA, and several interactive
effects. In \textbf{chapter 3}, this was followed up with a traditional
linear mixed modelling procedure where multiple row column designs with
various residual structures were tested on an augmented dataset
\autocite{Gilmour1997}. The BLUEs and standard errors from these models
where then used in a full genomic model where genomically estimated
BLUPs were evaluated in a cross-validation schema. While the spatial
modelling procedures from \textbf{chapter 2} and \textbf{chapter 3} were
derived from two different frameworks, we found little to no difference
between the use of semi-parametric Spline functions and structuring our
residual variance. Previous studies have also suggested these approaches
as equivalent with exception of the ease of a single model-fitting over
the testing of multiple linear mixed models \autocite{Velazco2017}. Both
modelling procedures from these chapters follow the general form of a
two-stage mixed model analysis where trials are independently analyzed
followed by an combined analysis on the genotypic means (most commonly)
together with weights on those genotypic observations
\autocite{Piepho2012}. Two-stage approaches are understood to be
generally less accurate than than their single stage counterparts
especially with greater model complexity \autocite{Gogel2018}. All this
being said, single-stage approaches tend to be unwieldy in their
specification and computation and the consequences of a two-stage
approach are unlikely to change any inferential conclusions from these
two chapters \autocite{Damesa2017}.

A core concept in hybrid breeding is the ability to assess the potential
of hybrid crosses on the basis of parental value. In the case for
phenotypic selection, general combining abilities tend to be the primary
mode which communicates a line's value \autocite{Bertan2007}. In
\textbf{chapter 2} we generated GCAs for 400 parental lines in multiple
tuber variates and found that the GCAs sufficiently captured a large
portion of the phenotypic variation in the F1 screening trials. The use
of GCA for selection is not unique to hybrid crops with many examples of
their use in tetraploid potato in many modern applications as well
\autocite{Neele1991,Bradshaw2003}. This is despite many tetraploid
studies reporting a large proportion of non-additive genetic effects
controlling tuber qualities like tuber size, tuber yield, and marketable
yield \autocites{Plaisted1962}[ ]{Tai1976}{Killick1977}. This suggests
an additional utility of diploid-based breeding with a simpler genetic
architecture in contrast to their polyploid counterparts
\autocite{Osborn2003}. This is evident even while examining a
single-locus model. Considering the allelic standard error of a locus
(\(\sigma_p = \sqrt{\frac{p\cdot (1-p)}{l \cdot N}}\)) in a population
of 10 individuals (\(N = 10\)) and an allele frequency of 0.35
(\(p = 0.35\)), the error would be 0.11 and 0.075 in a population of
diploids (\(l = 2\)) and tetraploids (\(l = 4\)), respectively. The
probability of a 0.1 positive shift in allele frequency
(\(P(p \geq 0.45) = 1 - \Phi\left(\frac{0.1}{\sigma_p}\right)\)) would
then be 0.17 in the diploid population versus 0.09 in the tetraploid
population. This is to demonstrate that gene frequencies are buffered
from perturbations making any directional selection in a tetraploid
breeding program slower than in their diploid counterpart. This impacts
many aspects of breeding whether targeting a single locus
(e.g.~marker-assisted selection) or attempting multi-locus selection
(e.g.~recurrent selection, selfing).

From \textbf{chapter 2}, some insight around the genetic architecture of
yield components was gleaned in the form of genetic correlations. This
was not just expressed in terms of a single genetic component over the
hybrids, but explicitly in terms of the GCA and SCA covariance
structures (\(\Sigma_G = 2 \cdot \Sigma_{gca} + \Sigma_{sca}\)). This is
worth re-emphasis. Potato has many traits are relevant for selection;
many of them being complex traits with low heritabilities and
interdependent relationships \autocite{Gebhardt2013}. Because of this,
it is critical for breeders to better utilize multivariate methods in
genetic improvement in potato. In addition, hybrid breeding schemas are
also augmented by multivariate methods whether based upon phenotype
alone or also for multivariate genomic prediction applications
\autocite{Wang2017}.

\subsection{Genomic prediction in hybrid
potato}\label{genomic-prediction-in-hybrid-potato}

Genomic prediction has emerged as a cornerstone of modern crop breeding,
offering unprecedented precision in estimating genetic merit by
leveraging genome-wide marker data. By integrating molecular markers
with phenotypic records, genomic prediction enables breeders to identify
superior parental combinations and hybrid crosses with greater accuracy
and efficiency than traditional pedigree-based methods. This section
explores the application, advantages, and limitations of genomic
prediction in hybrid potato breeding, drawing on insights from our
experimental chapters to highlight its transformative potential for
accelerating genetic gain.

Multiple chapters in this thesis addressed genomic prediction
applications in hybrid potato. In \textbf{chapter 3}, we built on the
partitioned GCA and SCA models from \textbf{chapter 2} and incorporated
molecular marker information to structure the each genetic component.
Multiple genomic models were then tested in a predictive application
testing a simple and full genetic model (GCA, and GCA+SCA, respectively)
for the same tuber variates from \textbf{chapter 2} along with tuber dry
matter content. Testing the predictive model performance between each
model showed no additional benefit through the addition of the SCA
component in the model. Contrasting the SCA variance relative to the
genetic residual suggested that there were other genetic effects that
were not captured in either of the other genetic effects, most notably
in total tuber number and dry matter content. Through this modelling
schema, we were able to confirm that genomic prediction solely on the
basis of GCA sufficed in the estimation of a hybrid cross genetic for
all tuber variates studied.

These prediction models were extended further in \textbf{chapter 5} by
examining several other statistical paradigms and assessing any benefits
to multiallelic marker information. We drew our attention specifically
to the utility of identity-by-descent (IBD) information derived from
deep pedigree information linking ancestral founders to the parent's of
hybrids along with multiallelic identity-by-state (IBS) information.
Both types of marker information were compared with conventional
biallelic SNPs using traditional shrinkage-based models along with more
complicated kernel prediction. For all tuber variates, the SNP based
prediction models were superior than their multiallelic counterparts.
Marked differences in prediction accuracy were especially observed
between the SNP and IBD models. Similarly, we found little differences
between the different modelling methods, with exception that the
Gaussian kernel tended to maximize prediction accuracy regardless of
trait and markerset. These results would suggest that in the context of
genomic prediction, simpler marker parameterizations tend to yield more
consistent prediction outcomes. Despite these results, multi-allelic
marker information are likely to continue to have a prominent role in
inferential applications such as multi-parental population (MPP) mapping
\autocite{Li2021a}. This utility will continue to advance potato
genetics \{TODO\}.

\subsection{Methods of selection}\label{methods-of-selection}

Modern breeding programs are met with unbounding choice with regard to
different technologies. This is especially true for molecular marker and
marker-based methods of selection. Marker-assisted selection (MAS) and
other marker-based techniques have revolutionized modern breeding
programs by enabling precise selection of desirable traits at early
stages of development. In the context of hybrid potato breeding, these
methods leverage molecular markers to identify and select parental lines
with favorable alleles, thereby accelerating genetic gain.

The topic of selection strategy in hybrid potato was tackled in
\textbf{chapter 4}. The primary exercise was to estimate the selection
efficiency of our marker-assisted selection strategy (MAS), a
linkage-based control (PC), and genome-wide prediction (GP) for the
prediction of an inbred parent's breeding value. Based upon a
forward-selection procedure, 33 unique QTL were found across three
traits (total tuber number was excluded). These QTL were able to pick up
between 54 and 56 per cent of the background genetic variation and were
the basis for prediction in the MAS models. The more efficient selection
schema was dependent on the statistic used for comparison. Examining
each strategy's selection accuracy, the GP strategy was evidently
superior for all three traits. However, when the accuracy of the
selection schema was controlled by the cost of the strategy, MAS had
sufficient selection accuracy per log cost.

\section{Wider Implications}\label{wider-implications}

\subsection{Hybrid Breeding Schema}\label{hybrid-breeding-schema}

Two important questions broached in \textbf{chapter's 2 \& 3} dealt with
the nature of gene action in hybrid potato and what population and
breeding strategy should be leveraged to effectively breed for complex
trait improvement. Both of these questions impact the future of what
\emph{kind} of hybrid crop potato might be. While the scope of this
thesis is limited by the limited genetic background sampled to create
these inbred populations \autocite{Lindhout2018}, we can still consider
how these results might inform strategy for potato breeders considering
hybrid schemas.

One of the major findings of the earlier chapters was a distinct lack of
SCA variance found among the panel of 806 hybrids from \textbf{chapter
2}. This was further confirmed in \textbf{chapter 3} in a genomic model
were the SCA variance was smaller than both the GCA and genetic residual
variance. As remarked in these chapters, this would indicate a lack of
non-additive gene action at work among our panel of F1 hybrids. This
could be characteristic of very little population structure among the
inbreds (as confirmed by the population-based analyses in
\textbf{chapter 3}), however, this is not wholly satisfactory. It has
been observed in other hybrid crops that SCA variance tends to be more
important in complex trait architecture where heterotic pools are not
genetically distinct \autocite{Zhao2015b}. Contrary to this, our results
follows Fisher's original observation where \(V_d\) was treated as
nothing more than a genetic residual in the parent offspring regression
\autocite{Bernardo2016}. This was evident both in the phenotypic
analyses and predictive modelling of \textbf{chapter's 2 \& 3}.

This naturally has ramifications in the structure of a breeding program.
Looking first at genomic prediction, we likely need not consider
higher-order genetic effects during model training. Models which either
considered the additive component (GCAs in \textbf{chapter 3}) or the
average genetic effect (the GW model in \textbf{chapter 4} and
shrinkage-based models in \textbf{chapter 5}) tended to be the most
reliable models for predicting hybrid performance. There is one caveat
here which is the increased performance observed with the Gaussian
kernel for the prediction of hybrid performance which marginally
outpaced the other approaches in \textbf{chapter 5}.

\subsection{Pooling Methods}\label{pooling-methods}

Perhaps most pertinent to hybrid breeding is with regard to pool
structure and the necessity of multiple pools. The development of pools
in crops like maize or sorghum was an unguided process where
complementation between distinct genetic groups was first observed often
with pedigree-breeding, and then further developed with more formal
methods of population improvement \autocite{Duvick2004}. Hybrid crops
are primarily the product of a multi-pool system, however, this is
dependent upon multiple conditions related both to complex trait
improvement and the additional costs of logistics in multi-pool
breeding. Multiple simulation studies have examined the topic of
heterotic pool development and are worth our consideration. Simulation
of both tetraploid and diploid breeding programs found that two pool
strategy based around GCA-based selections were effective for a clonal
diploid program, however, this was contingent on the degree of dominance
in the trait and is widely known to be capital intensive
\autocite{Labroo2023}. The question of how pools should be generated is
also open for debate. In terms of quantitative trait improvement,
multiple \emph{splitting} strategies have been suggested to be
equivalent including even random pool assignment, at least among selfing
crops \autocite{Cowling2020}. A more important consideration than
genetic differentiation among pools in potato will be the market segment
requirements. These will strongly govern pool development around trait
targets which widely diverge between ware tuber and processed potato
used in quick service restaurants as a matter of example
\autocite{Keijbets2008}. Rather than seeing heterotic pooling as some
metaphysical necessity, it is instead the working out of a comprehensive
strategy dependent on the affordable production of inbred lines, strong
fertility characteristics, and a reliable cytoplasmic male sterility
mechanism \autocite{McGrath2018}. These are the core factors which are
decisive to the choice of single, two, or multi-pool breeding schemas.
The topic of fertility is considered further in
Section~\ref{sec-future}.

\subsection{Multivariate applications for potato
breeders}\label{sec-implications}

Multivariate methods, including extensions like index selection, offer
transformative potential for crop breeding by enabling simultaneous
improvement of multiple traits while delineating their genetic,
environmental, and residual correlations. These approaches are
particularly valuable in potato breeding, where traits such as tuber
yield, quality parameters, and disease resistance often exhibit complex
interdependencies. By integrating genetic covariances, breeders can
design selection indices that maximize genetic gain across traits,
thereby optimizing resource allocation and accelerating progress toward
breeding objectives. While these methods have been widely adopted in
other field crops, their application in potato remains under-explored.

To lend some credence to multivariate selection in potato, we can
consider what multivariate selection would involve using the trait
genetic covariances for GCAs from \textbf{chapter 2} (Figure
\ref{fig:gca-coef-full-pairs}) together with their full phenotypic
variance matrix (\(\mathrm P\)). Assuming the GCA variances estimated
here would be roughly equivalent to those estimated from a test cross
(\(\mathrm {G} \approx \mathrm {V}_{gca}\)) schema in a hybrid breeding
program, we could estimate the selection response on inbred parents for
future inbred development using the multivariate breeder's equation as
expressed by \textcite{Lande1983}. If we performed truncation selection
on average tuber volume (\(s_{tv}~=~2~cm^3\)) while holding total tuber
yield and tuber number constant
(\(s_{ty}~=~0~Tonnes~\cdot~Ha^{-1}\);\textasciitilde{}\(s_{tn}~=~0~Tubers\)),
and conducted inter-mating between selected candidates, then the
expected selection response (\(\mathrm R\)) next test cross cycle could
be estimated as:

\[ \mathrm {R = G \cdot P^{-1} \cdot S}\]
\[ \mathrm {R} = \begin{bmatrix}11.54 & 7.27 & 4.95 \\ 7.27 & 11.13 & 67.2 \\ 4.95 & 67.2 & 629.51\end{bmatrix}\cdot\begin{bmatrix}0.17 & -0.22 & 0.02 \\ -0.22 & 0.34 & -0.03 \\ 0.02 & -0.03 & 0\end{bmatrix} \cdot \begin{bmatrix} 2 \\ 0 \\ 0 \end{bmatrix} \]
\[\mathrm {R} = \begin{bmatrix}1.09 \\ 0.64 \\ 0.07\end{bmatrix}\]

This would indicate an increase of about 1.1 \(cm^3\) in tuber volume
next test cross with a minor increase in total tuber yield and little
change in tuber number. Note that not only has this increased your
selection response when used over the univariate alternative (0.9
\(= h^2 s\)), but we are also able to evaluate the impact on the other
tuber variates. This is invaluable both in forecasting and breeding
strategy development.More attention is needed here to make these methods
more practical to wield as well as scalable in applied settings. With
regard to practicality, selection indices are one of the primary tools
used to reduce a breeder's multi-dimensional trait space into into a
singular index with a variety of methods proposed
\autocite{Kempthorne1959,Bulmer1981}. Speaking to implementability, the
most challenging step in this process is the estimation of the trait
genetic covariance. This tends to become quite unwieldy using
traditional linear mixed modeling methods in higher-dimensional spaces
making other methods more attractive \autocite{Blows2009,Runcie2013}.

This same exercise has utility in other facets of potato breeding. One
recurrent hurdle in potato evaluation is the lack of genetic correlation
between the initial seedling stages and subsequent clonal generations
\autocite{Maris1988}. This is true both for phenotypic evaluation and
genetic parameter estimation and all this significantly constrains early
selection efficiency within potato breeding programs
\autocite{Davies1974,Gopal1998}. Multivariate estimation of a
candidate's \emph{genetic value} (relevant in clonal programs) or
\emph{breeding value} (relevant for both clonal and hybrid breeding
schemas) jointly across multiple generations could be a valuable
extension of multivariate selection. With the genetic covariance of the
seedling and clonal generations tuber yield (for example), future clonal
generations could be predicted solely on the basis of a seedling's
yield. The estimation of these generational covariance structures
together with a technology like genomic prediction could be a valuable
extension in forecasting the potential of a candidate across
generations.

\subsection{Successful genomic prediction in hybrid
potato}\label{successful-genomic-prediction-in-hybrid-potato}

Genomic prediction has already accelerated genetic progress in multiple
hybrid crops \autocite{Labroo2021}. \textbf{Chapter's 3, 4, and 5} have
several implications for genomic prediction applications in potato as a
hybrid crop. Considering factors related to parameterization of genetic
model or type of modelling framework (what we will call more generally,
\emph{model choice}), it is not the \emph{decisive} factor in the
outcome of genomic prediction strategy. Having reviewed extensions of
the traditional GBLUP, various shrinkage-based estimation methods, and
one implementation of the Gaussian kernel, we found similar levels of
performance with only marginal improvement between models. This is in
keeping with many similar studies in other row crops as well as observed
in tetraploid potato
\autocite{Sverrisdottir2017,Endelman2018,Amadeu2019,Wilson2021a,Schrauf2021}.
Whether the model is an extension of GBLUP or a new member of the
Bayesian alphabet, there is no clear or distinct advantage observed for
most traits.

As documented in other crops, often more important than the specific
genetic parameterisations of a model is the actual composition of its
training set \autocite{Dias2019}. In many breeding programs, training
sets often arise from the breeding material itself utilizing previous
breeding cycles to estimate the population genetic covariance or marker
effects for a genomic model. Within hybrid breeding schemas, training
sets are frequently developed from test cross blocks which often have
some factorial structure (e.g.~\textbf{t} testers by \textbf{x} new
candidates) or sparse modification \autocite{Jarquin2020}. These are
used both for the selection of novel lines within-pools as well as the
scoping prospective crosses to aid intra and inter-pool improvement,
respectively {[}@{]}. \textbf{Chapter's 3, 4, and 5} all utilized four
field trials which utilized just under 800 F1 hybrids which were the
progeny of 456 inbred parents in a sparse mating design. While we were
able to demonstrate in an earlier chapter that the genetic models based
upon these training sets were technically \emph{identifiable}
\autocite{Xenakis2019}, the sparsity of the crossing block likely led to
bias in the estimation of our genetic variances and of any predictions
of the genetic effects \autocite{Mohring2011}. Despite this, our
training set still enabled genomic prediction of hybrid performance and
can likely be improved with more optimizations. Aside from incorporating
future cycles to augment the training set, there are many optimization
methods relevant to increasing the phenotypic and genetic variance in a
training set \autocite{Isidro2015,Berro2019,Ou2019}.

Related to training set composition is the topic of molecular marker
information density. These questions are often pursued along the lines
of predictive benefits in genomic modelling, the hope being that all
genetic variation can be interrogated by some perfect markerset. While
this cannot be fully achieved \autocite{Brachi2011}, markerset
composition is a worthwhile question in efficient selection
applications. In \textbf{Chapter's 4}, we considered a procedure for
comparing classical marker-assisted selection over genomic prediction
for several quantitative traits. One of the primary conclusions of this
research was that there are multiple complex traits in potato which
could be affectively selected for with only a handful of molecular
markers. This schema was relatively simple and could be augmented
through a more robust simulation of costs, genotype-by-environment
effects, and trial design \autocite{Peixoto2024}. A point should also be
made regarding our higher marker densities, that being, these were still
quite low. Many predictive modelling studies have used marker densities
in the magnitude of hundreds of thousands or million of molecular
markers \autocite{Druet2014,Weber2024}. Having said this, even
tetraploid studies have shown that small marker panels are still capable
of yielding permissible results in genomic prediction and other trait
discovery applications which is in keeping with what we observed
throughout this thesis \autocite{Aalborg2024,Leyva-Perez2022}.

Aside from marker density, molecular marker content has captured the
attention of many geneticists. Whether it be the inclusion of functional
genetic information, enrichment of major QTL, or multi-omic data, all
seek to include biologically relevant information to bolster genetic
applications \autocite{Guo2016,MacLeod2016,Arouisse2024}. In this vein,
\textbf{Chapter 5} attempted genomic prediction based upon multiallelic
predictors through the inclusion of haplotag and IBD profiles.
Strategies for the development of multiallelic probe design and genetic
models have become a topic of interest in potato as an answer to
potato's genomic diversity \autocite{Tang2022}. While recent pan-genome
studies have shown less haplotype diversity among released potato
varieties (particularly among European cultivars), potential
applications for multiallelic molecular data are still to be identified
\autocite{Sun2025}. Statistical models developed with our haplotag
probeset showed nearly identical performance with models using
traditional SNP-based predictors. These results would suggest little
benefit in the inclusion of multiallelic predictors for genome-wide
prediction based upon our survey of models used. The performance of the
IBD models were notably depressed relative to the other predictor
groups, but noted already, this could be the product of high uncertainty
for a subset of ancestral founders. Whether or not IBD information can
extend genomic prediction applications, it is worthwhile to reiterate
that founder composition in a breeding program can be utilized in
pre-breeding efforts, understanding selection outcomes, and genetic
diversity management.

\section{Future research}\label{sec-future}

There are many facets of potato breeding that have not been sufficiently
addressed by current quantitative genetic frameworks. We consider here
their role in improved fertility, pre-breeding methodology, and breeding
risk assessment for more robust potato breeding.

\subsection{Inbred Fertility}\label{inbred-fertility}

Fertility is a crucial factor for any seed-based crop. The success of
hybrid breeding systems are particularly dependent on seed cost price
which is a direct function of affordable and reliable seed production
systems \autocite{Mao1998,Longin2014}. Most research in potato fertility
up until the time of writing has been \emph{rightly} focussed on large
effect loci like \emph{Sli} or important genes in wider fertility
modules like \emph{StCDF1} \autocite{Clot2020,Eggers2021,Song2022}. As
these loci continue to be used and fixed in breeding populations, an
important future step will be to survey a broader array of relevant
traits diploid potato and assess them beyond QTL \autocite{Kempe2011}.
The typical targets for fertility in hybrid crops are pollen shed,
sufficient pollen viability, and synced male and female flower opening,
such that seed production is unencumbered
\autocite{Ortiz1997,Longin2012}. This in conjunction with the
development of key cytoplasmic male sterility (CMS) strategies will be
necessary to build a reliable seed production system in hybrid potato.

\subsection{Pre-breeding for Quantitative
Improvement}\label{pre-breeding-for-quantitative-improvement}

Pre-breeding is an often neglected topic in the complex trait
improvement of potato, especially in diploid-based programs. How novel
tetraploids should be screened and incorporated into a diploid breeding
program is not a straightforward process and is decorated by multiple
hazards. In traditional crop breeding, if a new program is being
initiated, evaluating a base population of landraces is a sensible trait
improvement strategy \autocite{Gorjanc2016}. However, in diploid
populations, novel tetraploid germplasm must first be subjected to
angiogenesis or \emph{prickle pollination} before dihaploids can be
introduced into a program \autocite{Uijtewaal1987,Tai2003}. This process
not only disrupts the original trait architecture of that original
tetraploid donor, but these dihaploids frequently bear little to no
resemblance to the original donor making their original evaluation
dubious.

Multiple screening approaches have been suggested, all of which, depend
on molecular marker information of some form. One approach proposed in
banana and potato suggests an extension of GBLUP whereby multi-ploidy
training sets could enable prediction from the 4x to 2x ploidy levels
\autocite{Nyine2018,Wilson2021}. The prediction accuracy observed in
these studies appear modest, but whether the endeavour of dedicated
tetraploid field evaluation for an augmented across-ploidy training set
is altogether justified is unclear. To avoid this, we could consider the
IBD estimation methodology used in \textbf{chapter 5} as a potential
method for linking the genetic value of dihaploids to their tetraploid
forbearer's. IBD estimation of deep pedigrees in diploids has been
realized in both simulated populations as well as real
\autocite{Li2021}. Additionally, IBD estimation has been conducted in
shallow structured populations in tetraploids
\autocite{Amadeu2021,Song2023}. Very little methodological development
would be needed to extend the IBD tracing from tetraploid to related
dihaploids enabling the estimation allelic effects (in diploid
populations) from specific tetraploid haplotypes. If a novel donor has
already been introgressed into a program, IBD profiles can leveraged and
even extended to related tetraploids.

Both these methods are essentially standard methods of \emph{forward
genetics}, where the performance of diploids is explained by some
molecular proxy in the tetraploid either through some genetic covariance
structure or via explicit IBD probabilities. One last consideration for
future improvements in pre-breeding is captured in what is often called
the next stage of breeding or \emph{``Breeding 4.0''}
\autocite{Wallace2018}. The central premise which has beguiled many a
R\&D manager is that a crop's \emph{ideotype} can be designed from the
ground up based upon a library of known haplotypes, functional
annotation, and gene-by-gene interactions which is then combined into an
optimal variety via gene-editing \autocite{Varshney2021}. Breeding is no
longer finding the proverbial needle in the haystack, but is instead
forged directly from its constituent components. Recent endeavours in
diploid potato breeding could be described as examples of crop design,
most notably, \textcite{Zhang2021}. Populations are screened on the
basis of functional genetic load based upon some algorithmic method, and
are then used to hasten the process of inbreeding. Whether this approach
can produce a sufficient voulme of inbreds to initialize a breeding
program is to be seen, but is a tool which is already being used with
the intent to design future potato cultivars.

\subsection{Formalization of Risk}\label{formalization-of-risk}

Often, while considering a novel technology or more efficient breeding
schema, the first question asked is \emph{``What kind of genetic gains
can be expected with this implementation''}? This in of itself is not
unsound, but without asking \emph{``what are the risks of such an
implementation''} biometricians run the risk of overly optimistic
forecasting. This addendum is especially pertinent for potato breeders
due to many of the aforementioned challenges, chief among them,
environmental sensitivity \autocite{George2017}. Formalizing the
estimation of explicit risk of a breeding strategy is the province of
quantitative genetics and thus deserves greater attention. For example,
decreasing the cycle length of a schema by reducing the number of years
of trial evaluation is a classic example for hastening genetic gain.
However, this would also increase the uncertainty around these
candidates if you have exposed them to an insufficient number of
environments. Embracing other tools might also bring risk to the
forefront. Factor analytic and random regression models have been lauded
for their ability to assess genotype stability and adaptability, and
offer incredible potential for evaluation in potato
\autocite{Souza2020,Smith2018}. Looking to methods utilized in the
animal breeding world, hazard models could also be useful for assessing
the long-term stability of a potential candidate \autocite{Yazdi2002a}.

\subsection{Conclusion}\label{conclusion}

Through this research, we hope to have provided a first examination into
the quantitative nature of several complex traits is hybrid potato.
Second, through the survey of a variety of statistical models, we

This thesis has demonstrated the transformative potential of biometrical
methods in advancing hybrid potato breeding. By interrogating the
genetic architecture of hybrid populations, evaluating genomic
prediction models, and assessing marker-based selection strategies, we
have provided a robust framework for accelerating genetic gain in
potato. Our findings underscore the importance of additive genetic
effects (GCAs) in hybrid performance prediction, the efficiency of
marker-assisted selection for major QTL, and the scalability of genomic
prediction even with modest marker densities. These insights not only
bridge critical gaps in potato breeding but also align with broader
trends in crop improvement, where quantitative genetics and genomic
technologies are reshaping traditional paradigms. Moving forward, the
integration of these methods into breeding programs promises to enhance
precision, reduce cycle times, and unlock the full potential of hybrid
potato as a sustainable crop for future agriculture.


\printbibliography



\end{document}
