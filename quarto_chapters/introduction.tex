% Options for packages loaded elsewhere
\PassOptionsToPackage{unicode}{hyperref}
\PassOptionsToPackage{hyphens}{url}
\PassOptionsToPackage{dvipsnames,svgnames,x11names}{xcolor}
%
\documentclass[
]{article}

\usepackage{amsmath,amssymb}
\usepackage{iftex}
\ifPDFTeX
  \usepackage[T1]{fontenc}
  \usepackage[utf8]{inputenc}
  \usepackage{textcomp} % provide euro and other symbols
\else % if luatex or xetex
  \usepackage{unicode-math}
  \defaultfontfeatures{Scale=MatchLowercase}
  \defaultfontfeatures[\rmfamily]{Ligatures=TeX,Scale=1}
\fi
\usepackage{lmodern}
\ifPDFTeX\else  
    % xetex/luatex font selection
\fi
% Use upquote if available, for straight quotes in verbatim environments
\IfFileExists{upquote.sty}{\usepackage{upquote}}{}
\IfFileExists{microtype.sty}{% use microtype if available
  \usepackage[]{microtype}
  \UseMicrotypeSet[protrusion]{basicmath} % disable protrusion for tt fonts
}{}
\makeatletter
\@ifundefined{KOMAClassName}{% if non-KOMA class
  \IfFileExists{parskip.sty}{%
    \usepackage{parskip}
  }{% else
    \setlength{\parindent}{0pt}
    \setlength{\parskip}{6pt plus 2pt minus 1pt}}
}{% if KOMA class
  \KOMAoptions{parskip=half}}
\makeatother
\usepackage{xcolor}
\usepackage[lmargin=20mm,rmargin=20mm]{geometry}
\setlength{\emergencystretch}{3em} % prevent overfull lines
\setcounter{secnumdepth}{5}
% Make \paragraph and \subparagraph free-standing
\makeatletter
\ifx\paragraph\undefined\else
  \let\oldparagraph\paragraph
  \renewcommand{\paragraph}{
    \@ifstar
      \xxxParagraphStar
      \xxxParagraphNoStar
  }
  \newcommand{\xxxParagraphStar}[1]{\oldparagraph*{#1}\mbox{}}
  \newcommand{\xxxParagraphNoStar}[1]{\oldparagraph{#1}\mbox{}}
\fi
\ifx\subparagraph\undefined\else
  \let\oldsubparagraph\subparagraph
  \renewcommand{\subparagraph}{
    \@ifstar
      \xxxSubParagraphStar
      \xxxSubParagraphNoStar
  }
  \newcommand{\xxxSubParagraphStar}[1]{\oldsubparagraph*{#1}\mbox{}}
  \newcommand{\xxxSubParagraphNoStar}[1]{\oldsubparagraph{#1}\mbox{}}
\fi
\makeatother


\providecommand{\tightlist}{%
  \setlength{\itemsep}{0pt}\setlength{\parskip}{0pt}}\usepackage{longtable,booktabs,array}
\usepackage{calc} % for calculating minipage widths
% Correct order of tables after \paragraph or \subparagraph
\usepackage{etoolbox}
\makeatletter
\patchcmd\longtable{\par}{\if@noskipsec\mbox{}\fi\par}{}{}
\makeatother
% Allow footnotes in longtable head/foot
\IfFileExists{footnotehyper.sty}{\usepackage{footnotehyper}}{\usepackage{footnote}}
\makesavenoteenv{longtable}
\usepackage{graphicx}
\makeatletter
\def\maxwidth{\ifdim\Gin@nat@width>\linewidth\linewidth\else\Gin@nat@width\fi}
\def\maxheight{\ifdim\Gin@nat@height>\textheight\textheight\else\Gin@nat@height\fi}
\makeatother
% Scale images if necessary, so that they will not overflow the page
% margins by default, and it is still possible to overwrite the defaults
% using explicit options in \includegraphics[width, height, ...]{}
\setkeys{Gin}{width=\maxwidth,height=\maxheight,keepaspectratio}
% Set default figure placement to htbp
\makeatletter
\def\fps@figure{htbp}
\makeatother

\usepackage{lineno}\linenumbers
\makeatletter
\@ifpackageloaded{caption}{}{\usepackage{caption}}
\AtBeginDocument{%
\ifdefined\contentsname
  \renewcommand*\contentsname{Table of contents}
\else
  \newcommand\contentsname{Table of contents}
\fi
\ifdefined\listfigurename
  \renewcommand*\listfigurename{List of Figures}
\else
  \newcommand\listfigurename{List of Figures}
\fi
\ifdefined\listtablename
  \renewcommand*\listtablename{List of Tables}
\else
  \newcommand\listtablename{List of Tables}
\fi
\ifdefined\figurename
  \renewcommand*\figurename{Figure}
\else
  \newcommand\figurename{Figure}
\fi
\ifdefined\tablename
  \renewcommand*\tablename{Table}
\else
  \newcommand\tablename{Table}
\fi
}
\@ifpackageloaded{float}{}{\usepackage{float}}
\floatstyle{ruled}
\@ifundefined{c@chapter}{\newfloat{codelisting}{h}{lop}}{\newfloat{codelisting}{h}{lop}[chapter]}
\floatname{codelisting}{Listing}
\newcommand*\listoflistings{\listof{codelisting}{List of Listings}}
\makeatother
\makeatletter
\makeatother
\makeatletter
\@ifpackageloaded{caption}{}{\usepackage{caption}}
\@ifpackageloaded{subcaption}{}{\usepackage{subcaption}}
\makeatother

\ifLuaTeX
  \usepackage{selnolig}  % disable illegal ligatures
\fi
\usepackage[style=authoryear,]{biblatex}
\addbibresource{../library.bib}
\usepackage{bookmark}

\IfFileExists{xurl.sty}{\usepackage{xurl}}{} % add URL line breaks if available
\urlstyle{same} % disable monospaced font for URLs
\hypersetup{
  pdftitle={General Introduction},
  colorlinks=true,
  linkcolor={blue},
  filecolor={Maroon},
  citecolor={Blue},
  urlcolor={Blue},
  pdfcreator={LaTeX via pandoc}}


\title{General Introduction}
\author{}
\date{}

\begin{document}
\maketitle


\subsection{Potato and potato
breeding}\label{potato-and-potato-breeding}

Modern potato exists as the culmination of millennium of domestication,
selection, and cultivation. Only over the past 100 years has

\begin{enumerate}
\def\labelenumi{\arabic{enumi}.}
\tightlist
\item
  Potato: The crop (2 Paragraphs)

  \begin{enumerate}
  \def\labelenumii{\roman{enumii})}
  \tightlist
  \item
    Overview context of the crop
  \item
    Evolutionary history
  \item
    Potato as a universal field crop
  \end{enumerate}
\item
  Breeding potato (2 Paragraphs)

  \begin{enumerate}
  \def\labelenumii{\roman{enumii})}
  \tightlist
  \item
    Initial hobby breeding
  \item
    Development of potato breeding into 20th century
  \end{enumerate}
\item
  Lack of genetic improvement in potato (2 Paragraphs)

  \begin{enumerate}
  \def\labelenumii{\roman{enumii})}
  \tightlist
  \item
    Less progress in potato relative to other crops
  \item
    Replication in early generations
  \item
    Impact of polyploidy
  \item
    Large selection surface
  \end{enumerate}
\item
  Potential Technologies for breeding potato (2 Paragraph)

  \begin{enumerate}
  \def\labelenumii{\roman{enumii})}
  \tightlist
  \item
    Discuss alternatives to clonal potato breeding
  \item
    Technologies such as MAS, QTL mapping, genomic selection
  \end{enumerate}
\end{enumerate}

\subsection{The application of Biometrics methods in potato
breeding}\label{the-application-of-biometrics-methods-in-potato-breeding}

\begin{enumerate}
\def\labelenumi{\arabic{enumi}.}
\setcounter{enumi}{4}
\tightlist
\item
  Genomic driven Hybrid breeding

  \begin{enumerate}
  \def\labelenumii{\roman{enumii})}
  \tightlist
  \item
    Conversion to hybrid breeding
  \item
    Seizing other tools and technologies used in hybrid breeding
  \item
    Utility of TPS based cropping systems and supply chains the
    incredible success of hybrid breeding \autocite{Crow1998}.
  \end{enumerate}
\item
  Statistical modelling needed to drive change in potato

  \begin{enumerate}
  \def\labelenumii{\roman{enumii})}
  \tightlist
  \item
    Conducting and evaluation of single-trials in potato
  \item
    Genotype data
  \item
    Phenotypic and genotypic data for large hybrid populations
  \end{enumerate}
\end{enumerate}

\subsection{Aim of this thesis}\label{aim-of-this-thesis}

With exception of a handful of published studies, there have been very
few biometrical inspections of the genetic architecture or survey of
genomic methods on hybrid potato {[}@{]}. This thesis hopes to do this
through two primary directives. First we wish to use the same
traditional statistical methodologies which were first utilized in other
hybrid crops to evaluate the performance of this panel of hybrid potato,
assess genetic mechanisms of multiple yield related phenotypes, and
\{\}. Second, we seek to use this same population to evaluate novel
modelling frameworks to understand what technologies are best wielded in
a hybrid breeding program for potato.

\textbf{Chapter 2} begins with the evaluation of an F1 hybrid multi
environmental trial. This chapter is concerned chiefly with the
estimation of intra-trial spatial variation and the statistical
delineation of several genetic components of variation, namely, additive
and non-additive genetic effects acting in this hybrid crossing block.
We consider what these genetic fractions in several tuber traits might
impact breeding.

\textbf{Chapter 3} examines the potential of multiple METs for the
purpose of training genomic prediction models. This is conducted on
several heritable traits with a specific examination of different
genetic parameterizations to assess prediction accuracy of each model.
The genomic models tested are analogous to those tested in
\textbf{chapter 2}, with the addition of a cross-validation testing
procedure.

\textbf{Chapter 4} takes a broader evaluation of genomic models and
evaluates their potential in several selection scenarios. Specifically,
the efficiency of genomic selection is contrasted against MAS selection
schemas based upon putative QTL in a panel of F1 hybrids. Quantitative
trait improvement in diploid hybrid potato.

\textbf{Chapter 5} considers the question of molecular marker
information and whether multiallelic marker data has any utility in
predictive applications. Two forms of multiallelic information are
considered: short range haplotypes (or haplotags) and founder-origin
variants (or identity-by-descent based variants). We consider the
utility of both forms of multiallelism in a broad survey of predictive
genomic models to assess their benefits over that of traditional
SNP-based predictors.

\textbf{Chapter 6} we conclude with a synthesis of the previous
experimental chapters, with a particular interest in addressing how
these chapters further biometrical applications in hybrid potato. We
conclude by offering a survey of currently under-served topics of
research and discuss their relevance to hybrid potato breeding systems.


\printbibliography



\end{document}
