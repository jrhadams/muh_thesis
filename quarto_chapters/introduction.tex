% Options for packages loaded elsewhere
\PassOptionsToPackage{unicode}{hyperref}
\PassOptionsToPackage{hyphens}{url}
\PassOptionsToPackage{dvipsnames,svgnames,x11names}{xcolor}
%
\documentclass[
]{article}

\usepackage{amsmath,amssymb}
\usepackage{iftex}
\ifPDFTeX
  \usepackage[T1]{fontenc}
  \usepackage[utf8]{inputenc}
  \usepackage{textcomp} % provide euro and other symbols
\else % if luatex or xetex
  \usepackage{unicode-math}
  \defaultfontfeatures{Scale=MatchLowercase}
  \defaultfontfeatures[\rmfamily]{Ligatures=TeX,Scale=1}
\fi
\usepackage{lmodern}
\ifPDFTeX\else  
    % xetex/luatex font selection
\fi
% Use upquote if available, for straight quotes in verbatim environments
\IfFileExists{upquote.sty}{\usepackage{upquote}}{}
\IfFileExists{microtype.sty}{% use microtype if available
  \usepackage[]{microtype}
  \UseMicrotypeSet[protrusion]{basicmath} % disable protrusion for tt fonts
}{}
\makeatletter
\@ifundefined{KOMAClassName}{% if non-KOMA class
  \IfFileExists{parskip.sty}{%
    \usepackage{parskip}
  }{% else
    \setlength{\parindent}{0pt}
    \setlength{\parskip}{6pt plus 2pt minus 1pt}}
}{% if KOMA class
  \KOMAoptions{parskip=half}}
\makeatother
\usepackage{xcolor}
\usepackage[lmargin=20mm,rmargin=20mm]{geometry}
\setlength{\emergencystretch}{3em} % prevent overfull lines
\setcounter{secnumdepth}{5}
% Make \paragraph and \subparagraph free-standing
\makeatletter
\ifx\paragraph\undefined\else
  \let\oldparagraph\paragraph
  \renewcommand{\paragraph}{
    \@ifstar
      \xxxParagraphStar
      \xxxParagraphNoStar
  }
  \newcommand{\xxxParagraphStar}[1]{\oldparagraph*{#1}\mbox{}}
  \newcommand{\xxxParagraphNoStar}[1]{\oldparagraph{#1}\mbox{}}
\fi
\ifx\subparagraph\undefined\else
  \let\oldsubparagraph\subparagraph
  \renewcommand{\subparagraph}{
    \@ifstar
      \xxxSubParagraphStar
      \xxxSubParagraphNoStar
  }
  \newcommand{\xxxSubParagraphStar}[1]{\oldsubparagraph*{#1}\mbox{}}
  \newcommand{\xxxSubParagraphNoStar}[1]{\oldsubparagraph{#1}\mbox{}}
\fi
\makeatother


\providecommand{\tightlist}{%
  \setlength{\itemsep}{0pt}\setlength{\parskip}{0pt}}\usepackage{longtable,booktabs,array}
\usepackage{calc} % for calculating minipage widths
% Correct order of tables after \paragraph or \subparagraph
\usepackage{etoolbox}
\makeatletter
\patchcmd\longtable{\par}{\if@noskipsec\mbox{}\fi\par}{}{}
\makeatother
% Allow footnotes in longtable head/foot
\IfFileExists{footnotehyper.sty}{\usepackage{footnotehyper}}{\usepackage{footnote}}
\makesavenoteenv{longtable}
\usepackage{graphicx}
\makeatletter
\def\maxwidth{\ifdim\Gin@nat@width>\linewidth\linewidth\else\Gin@nat@width\fi}
\def\maxheight{\ifdim\Gin@nat@height>\textheight\textheight\else\Gin@nat@height\fi}
\makeatother
% Scale images if necessary, so that they will not overflow the page
% margins by default, and it is still possible to overwrite the defaults
% using explicit options in \includegraphics[width, height, ...]{}
\setkeys{Gin}{width=\maxwidth,height=\maxheight,keepaspectratio}
% Set default figure placement to htbp
\makeatletter
\def\fps@figure{htbp}
\makeatother
% definitions for citeproc citations
\NewDocumentCommand\citeproctext{}{}
\NewDocumentCommand\citeproc{mm}{%
  \begingroup\def\citeproctext{#2}\cite{#1}\endgroup}
\makeatletter
 % allow citations to break across lines
 \let\@cite@ofmt\@firstofone
 % avoid brackets around text for \cite:
 \def\@biblabel#1{}
 \def\@cite#1#2{{#1\if@tempswa , #2\fi}}
\makeatother
\newlength{\cslhangindent}
\setlength{\cslhangindent}{1.5em}
\newlength{\csllabelwidth}
\setlength{\csllabelwidth}{3em}
\newenvironment{CSLReferences}[2] % #1 hanging-indent, #2 entry-spacing
 {\begin{list}{}{%
  \setlength{\itemindent}{0pt}
  \setlength{\leftmargin}{0pt}
  \setlength{\parsep}{0pt}
  % turn on hanging indent if param 1 is 1
  \ifodd #1
   \setlength{\leftmargin}{\cslhangindent}
   \setlength{\itemindent}{-1\cslhangindent}
  \fi
  % set entry spacing
  \setlength{\itemsep}{#2\baselineskip}}}
 {\end{list}}
\usepackage{calc}
\newcommand{\CSLBlock}[1]{\hfill\break\parbox[t]{\linewidth}{\strut\ignorespaces#1\strut}}
\newcommand{\CSLLeftMargin}[1]{\parbox[t]{\csllabelwidth}{\strut#1\strut}}
\newcommand{\CSLRightInline}[1]{\parbox[t]{\linewidth - \csllabelwidth}{\strut#1\strut}}
\newcommand{\CSLIndent}[1]{\hspace{\cslhangindent}#1}

\usepackage{lineno}\linenumbers
\makeatletter
\@ifpackageloaded{caption}{}{\usepackage{caption}}
\AtBeginDocument{%
\ifdefined\contentsname
  \renewcommand*\contentsname{Table of contents}
\else
  \newcommand\contentsname{Table of contents}
\fi
\ifdefined\listfigurename
  \renewcommand*\listfigurename{List of Figures}
\else
  \newcommand\listfigurename{List of Figures}
\fi
\ifdefined\listtablename
  \renewcommand*\listtablename{List of Tables}
\else
  \newcommand\listtablename{List of Tables}
\fi
\ifdefined\figurename
  \renewcommand*\figurename{Figure}
\else
  \newcommand\figurename{Figure}
\fi
\ifdefined\tablename
  \renewcommand*\tablename{Table}
\else
  \newcommand\tablename{Table}
\fi
}
\@ifpackageloaded{float}{}{\usepackage{float}}
\floatstyle{ruled}
\@ifundefined{c@chapter}{\newfloat{codelisting}{h}{lop}}{\newfloat{codelisting}{h}{lop}[chapter]}
\floatname{codelisting}{Listing}
\newcommand*\listoflistings{\listof{codelisting}{List of Listings}}
\makeatother
\makeatletter
\makeatother
\makeatletter
\@ifpackageloaded{caption}{}{\usepackage{caption}}
\@ifpackageloaded{subcaption}{}{\usepackage{subcaption}}
\makeatother

\ifLuaTeX
  \usepackage{selnolig}  % disable illegal ligatures
\fi
\usepackage{bookmark}

\IfFileExists{xurl.sty}{\usepackage{xurl}}{} % add URL line breaks if available
\urlstyle{same} % disable monospaced font for URLs
\hypersetup{
  pdftitle={General Introduction},
  colorlinks=true,
  linkcolor={blue},
  filecolor={Maroon},
  citecolor={Blue},
  urlcolor={Blue},
  pdfcreator={LaTeX via pandoc}}


\title{General Introduction}
\author{}
\date{}

\begin{document}
\maketitle


\section{Introduction}\label{introduction}

\subsection{Potato: The crop}\label{potato-the-crop}

Cultivated Potato (\emph{Solanum tuberosum}) is a food crop that has
become a staple for billions of people worldwide. Cultivated for its
starchy tubers, this vegetable crop has transformed the human diet and
contributed significantly to global food
security\textsuperscript{{[}1{]}}. Originating from modern-day Peru,
potato has a complex evolutionary history which is still under intense
scrutiny. Tuber bearing \emph{Solanum} species (known as \emph{Petota}),
represent one of the most diverse sections within the genus. Only
recently has the potato lineage been found to have originated as an
inter-specific hybrid between Etuberosum and tomato lineages upwards of
9 million years ago\textsuperscript{{[}2{]}}. Neither of these ancestral
species were tuber-bearing, and only through subsequent evolutionary
innovations did tuber formation emerge \{\}. This hybridization event
precipitated an unprecedented radiation with over 100 tuber-bearing
species distributed across the Andes stretching from modern-day Colombia
to Bolivia\textsuperscript{{[}3{]}}. This remarkable diversification was
possible through the incredible adaptation of these species to multiple
climates and ecological niches.

Domestication of potato is thought to have originated around 10,000
years ago, though still with many questions over its
origins\textsuperscript{{[}4{]}}. There are two origin theories, namely,
a single and multi-origin hypothesis for the domestication of potato.
The impact of domestication on potato did not only result in differences
in the harvested tubers (e.g.~enlargement of tubers reduction of
anti-nutrients content like glycoalkaloids), but
also\textsuperscript{{[}5{]}}. Through domestication, Through the
introduction of tetraploid Andean landraces to Europe in the
mid-16\textsuperscript{th} century, potato cultivation began to be
established in European agriculture. It was not until the introduction
and admixture with Chilean potato landraces that potato was adapted to
long-day conditions, leading to the widespread adoption and spread of
potato across Europe\textsuperscript{{[}6,7{]}}.

The origins of selective potato breeding in Europe, like many crops, was
quite rudimentary. Just as potato was becoming a universal staple in
England and Ireland in the mid-18th century, the growing of potato
seedlings was thought by some to to buffer the deterioration observed in
extant cultivars, perhaps alluding to the accumulation of pathogen in
seed tuber stocks\textsuperscript{{[}8{]}}.

\subsection{Modern Potato breeding}\label{modern-potato-breeding}

Many popular varieties in North America and Europe were produced between
1860 and 1900\textsuperscript{{[}9{]}}. This period of breeding was
especially prolific in potato's history saw the origination of important
clones such as Rough purple Chili
(\textasciitilde1850)\textsuperscript{{[}8{]}}. The progeny of these
clones are still used to the time of writing; these include varieties
such as Russet Burbank (1908), King Edward (1902), Bintje
(1910)\textsuperscript{{[}10{]}}.

\begin{enumerate}
\def\labelenumi{\arabic{enumi}.}
\setcounter{enumi}{2}
\tightlist
\item
  Lack of genetic improvement in potato (2 Paragraphs)

  \begin{enumerate}
  \def\labelenumii{\roman{enumii})}
  \tightlist
  \item
    Less progress in potato relative to other crops
  \item
    Replication in early generations
  \item
    Impact of polyploidy
  \item
    Large selection surface
  \end{enumerate}
\item
  Potential Technologies for breeding potato (2 Paragraph)

  \begin{enumerate}
  \def\labelenumii{\roman{enumii})}
  \tightlist
  \item
    Discuss alternatives to clonal potato breeding
  \item
    Technologies such as MAS, QTL mapping, genomic selection
  \end{enumerate}
\end{enumerate}

\subsection{Hybrid Potato breeding}\label{hybrid-potato-breeding}

By no means a new idea, adapting clonal tetraploid potato to a hybrid
breeding schema has been suggested as a method which could circumvent
many of the complexities of modern potato breeding. Some have even gone
so far as to claim that the stalled genetic progress in potato breeding
could be solved with breeding potato as a diploid. Efforts to breed with
diploids are almost as old as the breeding institutions in

\begin{itemize}
\tightlist
\item
  \textsuperscript{{[}11{]}} Speak about the breeding schema of Jansky
\end{itemize}

\begin{enumerate}
\def\labelenumi{\arabic{enumi}.}
\setcounter{enumi}{4}
\tightlist
\item
  Genomic driven Hybrid breeding

  \begin{enumerate}
  \def\labelenumii{\roman{enumii})}
  \tightlist
  \item
    Conversion to hybrid breeding
  \item
    Seizing other tools and technologies used in hybrid breeding
  \item
    Utility of TPS based cropping systems and supply chains the
    incredible success of hybrid breeding\textsuperscript{{[}12{]}}.
  \end{enumerate}
\end{enumerate}

The benefits of hybrid breeding in potato extend beyond the practical,
but also offer a greater ease of adoption of other techologies. Over the
past three decades, many the science and methodology around crop
improvement have rapidly developed, especially among field crops. This
is the product of

\begin{enumerate}
\def\labelenumi{\arabic{enumi}.}
\setcounter{enumi}{5}
\tightlist
\item
  Statistical modelling needed to drive change in potato

  \begin{enumerate}
  \def\labelenumii{\roman{enumii})}
  \tightlist
  \item
    Conducting and evaluation of single-trials in potato
  \item
    Genotype data
  \item
    Phenotypic and genotypic data for large hybrid populations
  \end{enumerate}
\end{enumerate}

The emergence of diploid hybrid programs offers a unique opportunity to
apply a particular subset of biometrical methodology and breeding
designs. Finally, this dataset is a useful snapshot of where

\subsection{Aim of this thesis}\label{aim-of-this-thesis}

With exception of a handful of published studies, there have been few
biometrical inspections of the genetic architecture or survey of genomic
methods on hybrid potato {[}@{]}. This thesis hopes to accomplish this
through two primary directives. First I wish to use the same traditional
statistical methodologies which were first utilized in other hybrid
crops to evaluate the performance of this panel of hybrid potato, assess
genetic mechanisms of multiple yield related phenotypes. Second, I seek
to use this same population to evaluate modern predictive modelling
frameworks to understand what technologies are best wielded in a hybrid
breeding program for potato.

\textbf{Chapter 2} begins with the evaluation of an F1 hybrid multi
environmental trial. This chapter is concerned chiefly with the
estimation of intra-trial spatial variation and the statistical
delineation of several genetic components of variation, namely, additive
and non-additive genetic effects acting in this hybrid crossing block. I
consider what these genetic fractions in several tuber traits might
impact breeding.

\textbf{Chapter 3} examines the potential of multiple METs for the
purpose of training genome-wide regression models. This is conducted on
several heritable traits with a specific examination of different
genetic parameterizations to assess prediction accuracy of each model.
The genomic models tested are analogous to those tested in
\textbf{chapter 2}, with the addition of a model validation procedure
designed for F1 hybrid datasets.

\textbf{Chapter 4} takes a broader evaluation of genomic models and
evaluates their potential in several selection scenarios. Specifically,
the efficiency of genomic selection is contrasted against MAS selection
schemas based upon putative QTL in a panel of F1 hybrids. The primary
aim was to evaluate whether quantitative trait improvement in diploid
hybrid potato could be more efficient with MAS based procedures.

\textbf{Chapter 5} considers the question of molecular marker
information and whether multiallelic marker data has any utility in
predictive applications. Two forms of multiallelic information are
considered: short range haplotypes (or haplotags) and founder-origin
variants (or identity-by-descent based variants). I consider the utility
of both forms of multiallelism in a broad survey of predictive genomic
models to assess their benefits over that of traditional SNP-based
predictors.

\textbf{Chapter 6} I conclude with a synthesis of the previous
experimental chapters, with a particular interest in addressing how
these chapters further biometrical applications in hybrid potato. We
conclude by offering a survey of currently under-served topics of
research and discuss their relevance to hybrid potato breeding systems.

\phantomsection\label{refs}
\begin{CSLReferences}{0}{1}
\bibitem[\citeproctext]{ref-Zaheer2016}
\CSLLeftMargin{1. }%
\CSLRightInline{Zaheer K, Akhtar MH. Potato {Production}, {Usage}, and
{Nutrition}---{A Review}. Critical Reviews in Food Science and Nutrition
{[}Internet{]} 2016 {[}cited 2025 Oct 19{]};56(5):711--21. Available
from: \url{https://doi.org/10.1080/10408398.2012.724479}}

\bibitem[\citeproctext]{ref-Zhang2025}
\CSLLeftMargin{2. }%
\CSLRightInline{Zhang Z, Zhang P, Ding Y, Wang Z, Ma Z, Gagnon E, et al.
Ancient hybridization underlies tuberization and radiation of the potato
lineage. Cell {[}Internet{]} 2025 {[}cited 2025 Oct
11{]};188(19):5249--5265.e15. Available from:
\url{https://www.cell.com/cell/abstract/S0092-8674(25)00736-6}}

\bibitem[\citeproctext]{ref-Spooner2014}
\CSLLeftMargin{3. }%
\CSLRightInline{Spooner DM, Ghislain M, Simon R, Jansky SH, Gavrilenko
T. \href{https://doi.org/10.1007/s12229-014-9146-y}{Systematics,
{Diversity}, {Genetics}, and {Evolution} of {Wild} and {Cultivated
Potatoes}}. Botanical Review 2014;80(4):283--383. }

\bibitem[\citeproctext]{ref-Spooner2005}
\CSLLeftMargin{4. }%
\CSLRightInline{Spooner DM, McLean K, Ramsay G, Waugh R, Bryan GJ.
\href{https://doi.org/10.1073/pnas.0507400102}{A single domestication
for potato based on multilocus amplified fragment length polymorphism
genotyping}. Proceedings of the National Academy of Sciences of the
United States of America 2005;102(41):14694--9. }

\bibitem[\citeproctext]{ref-Hardigan2017}
\CSLLeftMargin{5. }%
\CSLRightInline{Hardigan MA, Parker F, Laimbeer E, Newton L, Crisovan E,
Hamilton JP, et al. Genome diversity of tuber-bearing {Solanum} uncovers
complex evolutionary history and targets of domestication in the
cultivated potato. Proceedings of the National Academy of Sciences of
the United States of America {[}Internet{]} 2017 {[}cited 2019 Feb
14{]};114(46). Available from:
\href{https://www.pnas.org/cgi/doi/10.1073/pnas.1714380114}{www.pnas.org/cgi/doi/10.1073/pnas.1714380114}}

\bibitem[\citeproctext]{ref-DeJong2016}
\CSLLeftMargin{6. }%
\CSLRightInline{De Jong H. Impact of the {Potato} on {Society}. Am J
Potato Res {[}Internet{]} 2016 {[}cited 2025 Oct 11{]};93(5):415--29.
Available from: \url{https://doi.org/10.1007/s12230-016-9529-1}}

\bibitem[\citeproctext]{ref-Gutaker2019}
\CSLLeftMargin{7. }%
\CSLRightInline{Gutaker RM, Weiß CL, Ellis D, Anglin NL, Knapp S, Luis
Fernández-Alonso J, et al. The origins and adaptation of {European}
potatoes reconstructed from historical genomes. Nat Ecol Evol
{[}Internet{]} 2019 {[}cited 2025 Oct 11{]};3(7):1093--101. Available
from: \url{https://www.nature.com/articles/s41559-019-0921-3}}

\bibitem[\citeproctext]{ref-Glendinning1983a}
\CSLLeftMargin{8. }%
\CSLRightInline{Glendinning DR. Potato {Introductions} and {Breeding} up
to the {Early} 20th {Century}. New Phytologist {[}Internet{]} 1983
{[}cited 2025 Dec 6{]};94(3):479--505. Available from:
\url{https://onlinelibrary.wiley.com/doi/abs/10.1111/j.1469-8137.1983.tb03460.x}}

\bibitem[\citeproctext]{ref-Stevenson1937}
\CSLLeftMargin{9. }%
\CSLRightInline{Stevenson FJ, Clark CF. Breeding and {Genetics} in
{Potato Improvement} {[}Internet{]}. In: Yearbook of {Agriculture}. U.S.
Government Printing Office; 1937. page 405--44.Available from:
\url{https://books.google.com?id=gs9gQTvtpFMC}}

\bibitem[\citeproctext]{ref-VanBerloo2007}
\CSLLeftMargin{10. }%
\CSLRightInline{van Berloo R, Hutten RCB, van Eck HJ, Visser RGF. An
{Online Potato Pedigree Database Resource}. Potato Res {[}Internet{]}
2007 {[}cited 2025 Dec 6{]};50(1):45--57. Available from:
\url{https://doi.org/10.1007/s11540-007-9028-3}}

\bibitem[\citeproctext]{ref-Jansky2018}
\CSLLeftMargin{11. }%
\CSLRightInline{Jansky SH, Spooner DM. The {Evolution} of {Potato
Breeding} {[}Internet{]}. In: Goldman I, editor. Plant {Breeding
Reviews}. Wiley; 2018 {[}cited 2025 Dec 6{]}. page 169--214.Available
from:
\url{https://onlinelibrary.wiley.com/doi/10.1002/9781119414735.ch4}}

\bibitem[\citeproctext]{ref-Crow1998}
\CSLLeftMargin{12. }%
\CSLRightInline{Crow JF. 90 years ago: The beginning of hybrid maize.
Genetics {[}Internet{]} 1998 {[}cited 2025 Aug 9{]};148(3):923--8.
Available from:
\url{https://academic.oup.com/genetics/article-abstract/148/3/923/6034434}}

\end{CSLReferences}




\end{document}
